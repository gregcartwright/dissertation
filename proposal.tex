\documentclass[12pt,a4paper,oneside,titlepage]{article}
%titlepage prints the title on it's own page in article class

%doublespace my text
\renewcommand{\baselinestretch}{1.5}

%reduce the hyphenation of words
\sloppy

%prints a small space between paragraphs and removes the indenting of the first line
\setlength{\parskip}{2ex plus0.5ex minus 0.2ex}
\setlength{\parindent}{0em}

%csquotes - provides multilingual quoting - keeps babel happy
\usepackage{csquotes}
%babel required for biblatex to work. Specifying british last make it the language in use
%\usepackage[english,british]{babel}
\usepackage[british]{babel}

\usepackage[style=authoryear,backend=biber,
	giveninits=true,
	uniquename=init,
	language=british]{biblatex}

\defbibnote{needsfixing}{\emph{(this formatting needs looking at -- not currently to LSBU standards.
For example, need to include names of all the authors.)}}
% Add my bibliography file here
\addbibresource{references.bib}

%rotating allows text to be printed sideways
%multirow allows tables with cells spanning more than one row
% Remove this if not required
%\usepackage{rotating,multirow}


\begin{document}
\author{Gregory Cartwright}
\title{Benefits for of frailty assessment for community hospital inpatients.}
%\maketitle
\section*{Introduction}

\subsection*{The problem}
(Need to write something to introduce frailty)

(Need to talk about the local community hospital model)

The author has seen community hospital inpatients who are frail, get admitted to an acute hospital due to deterioration in their condition. Admission to hospital for a person with frailty carries significant risk; about 30\% of such patients will develop delirium,  and frailty is linked with significantly increased mortality \parencite{clegg:13, eeles:12}.


Sometimes, following retrospective review,  the deterioration could have been managed locally in the community hospital without subjecting the patient to an acute hospital admission. 
The level of frailty of these patients is not routinely assessed in the community hospital.
If this frailty level was known would it help guide decision making around planning for urgent treatment escalation for such patients?
Some patients in the community hospitals do have limits put on their escalation of treament. 

The \textcite{bgs:14} recommend that people with frailty should have a care plan developed that sets out what the patient wants to happen in the event of deterioration in their health, taking into account what is important to the patient.
It appears that frail patients in the community hospital setting should routinely have the potential benefits and risks of escalation of their treatment to an acute setting considered and discussed with them and their family.

\subsection*{Questions}
\begin{itemize}
\item Are we doing this without formally assessing their frailty?
\item Is there a link between the level of a patients frailty and whether or not urgent escalation of treament is considered?
\item Are there patients who are frail and do not have an urgent care plan developed when they should have?
\item If we formally assessed the frailty of patients in our care, would they get more appropriate treatment?
\end{itemize}

\section*{Literature review}
\section*{Aims and objectives}
\section*{Research design}

I aim to look retrospectively at patients notes and use the documentation of their comprehensive geriatric assessment (CGA) in the notes to give the patient a frailty socre using the clinical frailty scale (CFS). It is considered appropriate to use the CFS to assess frailty following a CGA. 
For those patients who I identify as being frail, I would like to look at whether an urgent care plan regarding escalation of their treatment to an acute setting has been considered during their stay.

\clearpage
\printbibliography[prenote=needsfixing]
\end{document}
