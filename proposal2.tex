\documentclass
[
	12pt,
	a4paper,
	oneside,
%	titlepage
]{article}
%titlepage prints the title on it's own page in article class

\pagestyle{myheadings}
\markright{Gregory Cartwright --- 3403548 --- dissertation proposal}
%doublespace my text
%\renewcommand{\baselinestretch}{1.5}

%reduce the hyphenation of words
\sloppy

%prints a small space between paragraphs and removes the indenting of the first line
\setlength{\parskip}{2ex plus0.5ex minus 0.2ex}
\setlength{\parindent}{0em}

%csquotes - provides multilingual quoting - keeps babel happy
\usepackage{csquotes}
%babel required for biblatex to work. Specifying british last make it the language in use
%\usepackage[english,british]{babel}
\usepackage[british]{babel}

\usepackage[style=authoryear,backend=biber,
	giveninits=true,
	dateabbrev=false,
	uniquename=init,
	citestyle=authoryear,
	dashed=false,
	maxcitenames=2,
	maxbibnames=99,
	sorting=nyt,
	language=british]{biblatex}

%Add a comma between name and year in citations
\renewcommand*{\nameyeardelim}{\addcomma\space}
%put double space between entries in the references
\setlength\bibitemsep{2\itemsep}

%remove hanging indent from the reference list
\setlength\bibhang{0em}

%Ensure all names in reference list are formatted as Surname, I.
\DeclareNameAlias{default}{last-first}
\DeclareNameAlias{sortname}{last-first}

%Code to format journal article volume and issue as v (i)
\renewbibmacro*{volume+number+eid}{
	\printfield{volume}
	\setunit*{\addnbspace}
	\printfield{number}
	\setunit{\addcomma\space}
	\printfield{eid}
}
\DeclareFieldFormat[article]{number}{\mkbibparens{#1}}

%Code to format the 'Accessed day month year' in the references
\DeclareFieldFormat{urldate}{%
	[Accessed \thefield{urlday}\addspace%
  \mkbibmonth{\thefield{urlmonth}}\addspace%
  \thefield{urlyear}]}
  
%Code to format the 'Available from:' in the URL in references
\DeclareFieldFormat{url}{\bibstring{urlfrom}\addcolon\space\url{#1}}
\DefineBibliographyStrings{british}{
	urlfrom = {Available from}
}

%Command to STOP the references section and everything after having it's own header.
\defbibheading{bibliography}[\refname]{\section*{#1}}

\usepackage{graphicx}

% Add my bibliography file here
\addbibresource{references.bib}

%rotating allows text to be printed sideways
%multirow allows tables with cells spanning more than one row
% Remove this if not required
%\usepackage{rotating,multirow}


\begin{document}
\author{Gregory Cartwright\\
	Student number 3403548\\
	MSc Advanced Nurse Practitioner dissertation proposal
}
\title{Does level of frailty influence advance care planning 
	for community hospital inpatients?
}
\maketitle
\begin{abstract}
Community hospital patients generally have multiple co-morbidities and varying
levels of dependence. The prevalence of frailty is therefore likely to be high.
Whilst these patients are usually clinically stable when admitted, their health
can deteriorate. This deterioration can often be addressed in the community hospital
but sometimes for optimal management it is necessary for the patient's treatment to be
escalated to an acute setting. The risks of such transfers for people with frailty are high
and sometimes outweigh the benefits, and outcomes may be better if the transfer
does not happen. This is not always easily appreciated during the out of hours period.
Patients with frailty should be involved in discussions about
potential deterioration and advance care planning should be undertaken.

This study aims to retrospectively evaluate the care of 100 community hospital
patients to assess if there is an association between the level of frailty and other
factors (age, gender, presenting complaint) and advance
care planning and to provide guidance for future practice to 
help avoid unhelpful escalation of treatment and improve patient outcomes and 
experience.

A convenience sampling method will be used to select recently discharged patients.
Data will be collected from reviewing patient
case-notes. The data will be presented graphically and non-parametric tests, 
Spearman's rank correlation and chi-squared or Fisher's exact test, will 
be used to look for association between the patient factors and advance care planning.
\end{abstract}
\section{Introduction}

The population of the UK is getting older and this trend is forecast to continue.
In 2016 18\% of the population was over 65 years old, with 2.4\% over 85 years old.
Both these figures are expected to grow over the next 20 years \parencite{ons:17}.
Frailty is widely agreed to be a condition where the maintenance of homoeostasis 
becomes vulnerable to 
small stressors \parencite{vellas:16}.  Examples of such stressors include changes in environment and minor
illness. The consequences of exposure to these include delirium, significant reduction in mobility,
falls, increased dependency, non-specific failure to thrive and death 
\parencite{bgs:14,oliver:14,vellas:16}.
The prevalence of frailty amongst the ageing population is high \parencite{clegg:13},
with about 10\% of those aged over 65 and up to 50\% of
those aged over 85\% having frailty \parencite{bgs:14}. These figures
are thought to be higher still in the group of people that are housebound, 
have care at home or live in a care home \parencite{oliver:14}.

The author is an advanced nurse practitioner (ANP) in a community Trust.
The Trust has community hospitals which have wards for inpatient rehabilitation and
medical step-down. There are twelve wards across eight community hospitals. Each ward has
an ANP who works on the ward to provide medical management of the patients during 
the hours of Monday to Friday, 0900 to 1730. Outside of these hours, medical care 
is provided by the out of hours (OOH) general practitioner (GP) service. 

Most patients are admitted to the community hospital wards for either rehabilitation
or medical step-down. The majority of these patients come following an acute admission
where they have been stabilised medically but are often deconditioned as a result
of acute illness and are not ready to go home. At this stage their needs include 
ongoing medical treatment and monitoring and further assessments and treatment such 
as physiotherapy and occupational therapy to prepare them for discharge home.

Some patients are admitted directly from home because they have medical or rehabilitation
needs that cannot be met at home but don't require an admission to an acute hospital.
A minority of patients are admitted for palliative care.

When patients are admitted to one of these wards they have a comprehensive geriatric 
assessment (CGA) \parencite{bgs:14}. This is a multi-faceted diagnostic process
performed by the multi-disciplinary team (MDT) to produce a plan for treatment 
and follow-up.
An international meta-analysis found that, when compared with general medical care,
CGA was effective at keeping older people alive and living in their own homes at
twelve months post admission with a number needed to treat of 33 \parencite{ellis:11}.

The author expects that many of the patients admitted to the wards have frailty.
The numbers of patients with frailty is not known, however the patients are assessed 
for frailty and the patients admitted are old. During the last financial year the
average age was 81 and 49\% of patients admitted were aged 85 or over. Many have 
care needs as seen above.

Although the patients are usually clinically stable when they arrive on the ward,
their condition can deteriorate. This is usually managed in the community hospital
ward environment. The patients can have investigations including blood tests, plain
x-rays and electrocardiograms (ECG) usually without leaving the community hospital.
They can also receive treatments such as intravenous (IV) fluids, IV antibiotics
and even blood transfusions. 

Sometimes the deterioration is such that, 
for optimal management, services are required that can only be offered in secondary 
care. When this is the case a decision has to be made, weighing up the proposed
and likely benefits of transferring the patient as an urgent or emergency case to
an acute hospital with the risks that this presents to the patient. During the in-hours
period this decision is made by the ANP who may liaise with the consultant geriatrician. During
the OOH period the nurses make a telephone referral to the OOH GP.

This decision making is difficult for the OOH GP who usually does not know the patient.
Also the patient is acutely unwell and may not be able to participate in discussions
about their care. If it is nighttime it may be difficult to contact relatives to 
have such discussions. In these circumstances it could
be argued that the safest option is to admit the patient to the emergency department (ED) where their 
condition can be further assessed and closely monitored.

There is evidence that patients with frailty often have poor outcomes, including death,
following admission to acute hospital \parencite{silver:12, wallis:15}. 
This is recognised by the ANPs and for some frail patients they 
will have discussions with the patient and their family about the relative risks
and benefits of a potential admission to an acute hospital and what they would want to 
happen in the event of deterioration in their health. This way a person-centred 
advance care plan (ACP) can be made before any deterioration happens.

When patients deteriorate during the OOH period and the patient
gets sent to the acute Trust, the ANP retrospectively reviews what happened to see if the admission
could have been avoided. There are times when the ANP feels that, in those particular
circumstances, the risks to the patient of the acute admission outweigh the benefits
due to their frailty, and that the patient should have had an ACP
to avoid this acute admission. Some patients have an ACP made before admission, 
either in the community or the acute Trust,
and there are also patients who decline the process of advance care planning. It appears,
however, that there are patients for whom an ACP is not considered when it really should be.

There are national guidelines for advance care planning \parencite{rcp:09}. 
Practice in the local health community varies between settings and a multi agency 
group is currently reviewing this with the aim of producing guidelines that can 
applied in primary, secondary and intermediate care. The national guidance is not specific
about what factors should trigger advance care planning, saying that such a decision
should be based on clinical judgement.

In the community hospital wards the level of frailty of patients is recorded but 
not formally used to prompt advance
care planning. This project aims to examine if there is an association between the 
level of frailty in these community hospital inpatients and whether advance care planning
is undertaken. There may be other factors that influence this decision and the
study will also consider whether patient age, gender or presenting complaint have 
a relationship with advance care planning.

\section{Literature review}

A literature search was performed using the London South Bank University Library 
online catalogue. The search engines used were CINAHL Complete and MEDLINE. The
search terms used were ``frailty assessment", ``frailty outcomes" and ``frailty 
interventions".

\subsection{Assessing frailty}
An international systematic review found 
that at least 10\% of people aged over 65 had frailty and of those aged 85 or over
at least 26\% were frail \parencite{collard:12}. There are various methods and tools used to assess frailty.
Some of these count the number of deficits from a particular set that a person has. 
\textcite{sternberg:08} suggest that such a tool is not practical for use in a clinical
setting, being more suited to assessing populations for strategic planning. 
Other tools require numerous specific measurements to be taken, again making them
less suitable for clinical use \parencite{martin:08} and possibly more suited to research purposes
\parencite{ensrud:08}. \textcite{romero-ortuno:16} argue that this fragmentation should not
be viewed as a problem as each frailty assessment tool is suited to a different 
purpose. 

In both the clinical area which the author works and the ED at
the local acute Trust, frailty is assessed using the Clinical Frailty Scale (CFS,
see appendix~\ref{appendix:CFS}).
The CFS is a tool that has been validated for use in clinical practice 
\parencite{rockwood:05}. It rates frailty based on the person's level of independence
and dependence, giving them an ordinal position on the continuum from very fit and completely 
independent, CFS of 1, to very severely frail and completely dependent, CFS of 8.
There is also a CFS of 9 for those who are terminally ill. The CFS is based on clinical
judgement of the patient and is therefore suited to clinical use, certainly after
the patient has had a CGA \parencite{bgs:14}.

Generally, the majority of the population of community hospitals is frail older 
people \parencite{silver:12}.
The CFS score has recently been introduced to the community hospital wards in the author's Trust. The
score is not being used for any specific purpose, it is just being entered into 
the patients' notes for people to refer to. These scores are not being collected, 
but the author suspects that the proportion of inpatients who are at least 
moderately frail, with CFS at least 6, will be quite high. Specifically because
CFS is partly based on a person's ability to carry out activities of daily living (ADLs)
and instrumental activities of daily living (IADLs) and one reason that many patients
are admitted to the community hospital is that they need rehabilitation to be
able to carry out ADLs and IADLs. Indeed in a study to assess the prevalence of frailty
in France, \textcite{cossec:16} found that dependency on others for IADLs was an 
independent determinant of frailty.

\subsection{Consequences of frailty}

By definition, frailty is a state where a small change, intrinsic or extrinsic, can
lead to multiple consequences \parencite{collard:12}. These can be severe, including 
death. An examination
of all bereavements of adults in England that were not due to accident, homicide or suicide for 
a four month period in 2012 was 
carried out \parencite{ons:13}. It found that the proportion of deaths
that were not due to cancer or any cardiovascular disease (CVD) was 42\%. In the over 80
age group this was 80\%. The most recent edition of this survey from 2015 found
that the overall proportion of non-cancer and non-CVD deaths was slightly higher 
at 46\% \parencite{ons:16}, but did not provide a breakdown by age group. They 
did however report that 60\% of their sample were aged over 80.

How many of these deaths were due to frailty is not known, however 
a Canadian study that examined all deaths in Alberta found that frailty was the
cause of 30\% of mortality \parencite{fassbender:09}.

A national guide for emergency and urgent care of older people \parencite{silver:12}
reports that many older people are admitted to hospital only to die within hours.
This is particularly true for people admitted during the out of hours period.
Whilst this does not quantify the sequelae of frailty, there is literature that 
supports this. \textcite{wallis:15} performed a retrospective study looking 
at outcomes of hospital admission for people aged 75 and over, in relation to CFS. 
They found that increased frailty was an independent predictor of both 30-day
readmission and inpatient death, with nearly a quarter of people with CFS of 8 dying 
during the admission. This supports the work of \textcite{kang:15} whose Chinese 
study also found that frailty was independently associated with an increased risk of 
inpatient death, and also significantly increased the risk of readmission and 
3-month mortality for those who survived to hospital discharge.

The effect of frailty on those discharged from hospital was also studied by 
\parencite{kahlon:15}, who also found that frailty significantly increased the 
risk of both readmission and death within 30 days of discharge.

\subsection{Interventions for frailty}
Frailty is clearly a problem, so what should be done about it? Older patients 
should be screened for frailty at every contact with a health professional 
\parencite{bgs:14}, and 
this is already happening in the community hospital wards. There is a consensus that 
frailty should be viewed as a syndrome with multiple domains and therefore
there should be an MDT approach to it's management \parencite{vellas:16}.

The CGA is a multidisciplinary, evidence based strategy to guide the management 
of frailty that is viewed as best practice \parencite{silver:12, bgs:14, oliver:14}. Patients
in the community hospitals already receive this regardless of their CFS score.

We have seen how frailty combined with acute illness carries a high risk of death, and 
\textcite{silver:12} reports that end of life care in older people with frailty
is something that is often not adequately considered. \textcite{oliver:14} support this
by asserting that people with frailty are often not involved in planning their 
end-of-life care. They suggest that the reasons for this include factors such as
the trajectory, with frailty often being a more gradual decline without sudden 
landmark moments. This contrasts with conditions such as terminal cancer where there 
is a defining transition to an end-of-life phase. This can mean that entering such a
phase is not recognised and therefore planning is overlooked. 

Should the recognition of a high level of frailty act as an event to signify
that a person is entering an end-of-life phase? Having found that frailty in the 
context of acute coronary syndrome is associated
with poor outcomes, \textcite{kang:15} recommend that a high CFS should
trigger the consideration of escalation pathways \parencite{kang:15}.
This supports the recommendations of \textcite{silver:12} who assert that over-investigation
and unnecessary interventions in the frail elderly population are costly to both the
individual and the health economy. They go on to advise that in such patients, 
their preferences for their future care should be ascertained early. \textcite{oliver:14} 
reinforce this by highlighting the importance of gathering this information
before the person loses the capacity to make decisions about how their care should
progress. The later work of \textcite{romero-ortuno:16} adds weight to this argument.
Having identified the increased risk associated with frailty and hospital admission, 
they recommend that frailty generally should trigger personalised planning of
care and it's escalation.

\section{Aim and objectives}

\subsection{Aim}
The overall aim of this study will be to examine if there is a relationship between
level of frailty of community hospital in-patients and whether advance care planning
happens and what other factors might influence this process.

\subsection{Objectives}

\begin{enumerate}
\item	Ascertain the prevalence of different levels of frailty within the local community
		hospital population, and within these levels identify how many do not have
		an ACP before admission.\label{obj:prevalence}
\item	Examine the relationship between frailty and other patient factors and
		whether advance care planning is undertaken.\label{obj:association}
\item	Formulate local recommendations for practice to help reduce unhelpful
		acute hospital admissions for people with frailty through more effective
		preemptive planning of treatment escalation.
\end{enumerate}

\section{Study design} 
\label{sec:design}
The objectives require collection of numbers
and proportions of patients that meet objective criteria. A quantitative design
and a positivist methodology will be appropriate to this approach \parencite{parahoo:14}.

The study aims to examine the association between variables of 
the level of frailty and whether advance care planning
was carried out. It therefore seems appropriate to use a correlational design. 
It will be a retrospective observational 
cross-sectional study: the case-notes of discharged patients will be reviewed. To achieve 
objective~\ref{obj:prevalence}
data will be obtained by reviewing the case-notes
of patients, examining the initial MDT assessments to ascertain CFS score and
whether an advance care plan was in place prior to admission. 
For objective~\ref{obj:association} the entire case-notes of patients 
will be reviewed to ascertain whether advance care planning 
was considered during the stay and to obtain other information that may have influenced
advance care planning.

\section{Sample}
Time for data collection is limited due to the timescale for the dissertation. 
Therefore the sampling method needs to capture as many patients as possible in a 
short time. To facilitate this a convenience sampling method will be used.

An electronic patient record (EPR) is currently being rolled out across the Trust.
Currently eight out of the twelve wards have this implemented, so all the patient 
records for patients on these wards are accessible by the author remotely. This
accounts for 148 of the total of 214 beds: 69\%. 

The average length of stay (LOS) is 20.4 days and there are approximately 220
discharges each month across all wards. Of these a sample of 100 patients 
will be obtained by including the first 100
patients discharged from EPR wards on or after 1 December 2017. A limitation
of this sampling method is that it only includes people who had a hospital stay
during the winter period and thus may not be representative of the other seasons.

\section{Research instrument}

To achieve the objectives data will have to be collected. For each patient it 
will be necessary to capture whether they have a pre-existing
ACP and whether advance care planning happens whilst they are in the community
hospital. The factors being considered as possibly associated with the decision to
undertake advance care planning (age, gender, CFS and presenting complaint) will
also need to be recorded. If a patient has declined advance care planning this will
also need to be captured. A proposed data collection tool is provided
(see Appendix~\ref{appendix:tool}).

\section{Procedure}
\label{sec:procedure}
For each member of the sample, the EPR will be reviewed. 
For objective \ref{obj:prevalence} 
the initial admission and clerking documentation will be reviewed to obtain and
record the CFS score and the other relevant data. This will be recorded in the tool.


To collect data for objective \ref{obj:association}, all patients who did not have
a pre-existing plan when they were admitted will then have their EPR searched
for the duration of their admission. The record will be searched for the following
terms:

\begin{itemize}
\item acute
\item escalation
\item advance care plan
\item ACP
\item deterioration
\end{itemize}

Where one of these terms is found, the record will be read in context to assess if 
escalation planning was being considered. If advance care planning 
was considered at least once during the admission then this will be recorded as ``YES"
in the data collection tool. Otherwise ``NO" will be recorded in that column.

\section{Data analysis}
As discussed above, the author expects that there will be a high prevalence of frailty 
within the community hospital population. Therefore it is unlikely that the frequency
of frailty scores will be normally distributed. To achieve objective~\ref{obj:prevalence}
a bar chart will be produced to demonstrate the frequency of patients with each CFS.
Frailty scores will also be grouped together into no frailty, mild frailty (CFS 1--5),
moderate frailty (CFS 6) and severe frailty (CFS $\geq$ 7). A bar chart will also 
be produced to show the numbers and percentages that do not have an ACP on admission
for each CFS score and CFS group: none, mild, moderate or severe.

For objective~\ref{obj:association} the data will initially be presented in graphs.
Bar charts will be used to show the relative frequency with which advance care planning
was carried out for each level of CFS and group of frailty. A histogram will be 
used to show relative frequencies of ACP by age group. Bar charts 
will be used to shows the corresponding relative frequencies for gender and initial
presenting complaint. Population pyramids will also be used to look at these relative
frequencies for age and level of frailty allowing the genders to be compared.

Due to the data being ordinal and nominal in nature, non-parametric measures will
be used to look for correlations. A Spearman's rank correlation will be performed
to assess a relationship between level of frailty and the presence of an ACP. To do
this frailty will be ranked as 1=none, 2=mild, 3=moderate, 4=severe. The presence
of an ACP will be ranked as 1=does not have ACP, 2=has ACP.

To further investigate association between frailty and the other variables and advance
care planning, a non-parametric test such as a chi-squared test or a Fisher's exact
test will be applied. Using these tests statistical significance will be taken
to be a p-value of $p<0.05$.

Statistical analysis will be performed using Statistical Package for the Social 
Sciences (SPSS) version 21.

\section{Ethical considerations}

To ensure that there is appropriate approval from the Trust to undertake this project
the author contacted their line manager, the Trust Head of Research and Development and
the Trust Caldicott Guardian and sent them copies of this document. 

The line manager has given their full support to the project (see Appendix~\ref{appendix:caroline}).
The head of research and development
advised that the project sounded like it was probably not research and that 
the Health Research Authority (HRA) online decision tool should be used to confirm this.
They advised that if this tool says that the study is not classed as research then
the Trust Clinical Governance group should be contacted. The HRA tool did show
that the study is not considered to be research (see Appendix~\ref{appendix:hra}).
The Trust Clinical Governance Lead was contacted
and has given her approval to the project. This is subject to formal approval
by the Trust Clinical Governance and Effectiveness Audit group who will review 
the proposal at their next meeting on 5 February 2018 (see Appendix~\ref{appendix:clingov}).

The Trust Caldicott Guardian has advised that there should be further discussion with 
the research team prior to his final approval. This has taken place and their 
approval has been granted (see Appendix~\ref{appendix:daveclarke}).

\textcite{biggam:15} identifies five ethical principles that a study should maintain:
\begin{enumerate}
\item Do no harm
\item Impartial
\item Transparent
\item Confidential
\item Voluntary
\end{enumerate}

\subsection{Do no harm}
This study will not influence the care or treatment of any current patients therefore
there is no possibility of it causing harm to patients.

\subsection{Transparent}
This document sets out clearly how the study will be conducted.

\subsection{Impartial}
The author is an employee of the Trust where the study will take place. If any bad
practice is discovered as part of the work undertaken then the relevant manager
will be informed. Although the author usually works in the community hospital wards,
since December 2017 he has been undertaking a rotation into the acute Trust. This 
means that he will not have been involved in the community hospital care of any
patients involved in the study, reducing the risk of bias.

It is not expected that the author is a relative of any of the patients
whose notes will be examined as part of the study. If it transpires that such a 
relationship exists then that patient will be excluded from the study to eliminate
any possible resulting bias.

\subsection{Confidential}
To ensure that the confidentiality of patients is maintained during this study there
will be no recording of patient identifying data outside of the EPR. Each patient
record will be allocated a study ID number which will be recorded in the data collection
tool (see Appendix~\ref{appendix:tool}). This will anonymise the records.

The data collection tool will be an electronic Microsoft Excel 2010 spreadsheet
which will be stored on a Trust owned laptop which is encrypted and password protected. A backup 
of the data will be maintained on a secure Trust server which will be accessed over
the secure Trust network. All the data will be destroyed two years following completion 
of the study; the end of May 2020.

\subsection{Voluntary}

This project looks at retrospective data following patient discharge to evaluate 
an aspect of patient management. As such, this criteria is not relevant.

\section{Timetable}
Here is the proposed timetable for completion of this dissertation:
\includegraphics[width=\textwidth]{DissertationSchedule}

\section{Planned style of dissertation}
A traditional style of dissertation is planned.

\clearpage
\printbibliography

\clearpage
\begin{appendix}

\section{Clinical Frailty Scale}
\label{appendix:CFS}
\includegraphics[width=\textwidth]{CFS}

\section{Data collection tool}
\label{appendix:tool}
\includegraphics[width=\textwidth]{dataCollection}

\section{Line manager approval}
\label{appendix:caroline}
\includegraphics[width=\textwidth]{caroline.pdf}

\section{Health Research Authority decision tool result}
\label{appendix:hra}
\includegraphics[width=\textwidth]{hraresult.pdf}

\section{Clinical Governance Lead response}
\label{appendix:clingov}
\includegraphics[width=\textwidth]{heather.pdf}

\section{Research and Development response}
\label{appendix:daveclarke}
\includegraphics[width=\textwidth]{daveclarke}

\end{appendix}

\end{document}
