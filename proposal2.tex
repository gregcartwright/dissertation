\documentclass[12pt,a4paper,oneside,titlepage]{article}
%titlepage prints the title on it's own page in article class

%doublespace my text
\renewcommand{\baselinestretch}{1.5}

%reduce the hyphenation of words
\sloppy

%prints a small space between paragraphs and removes the indenting of the first line
\setlength{\parskip}{2ex plus0.5ex minus 0.2ex}
\setlength{\parindent}{0em}

%csquotes - provides multilingual quoting - keeps babel happy
\usepackage{csquotes}
%babel required for biblatex to work. Specifying british last make it the language in use
%\usepackage[english,british]{babel}
\usepackage[british]{babel}

\usepackage[style=authoryear,backend=biber,
	giveninits=true,
	uniquename=init,
	language=british]{biblatex}

\defbibnote{needsfixing}{\emph{(this formatting needs looking at -- not currently to LSBU standards.
For example, need to include names of all the authors.)}}
% Add my bibliography file here
\addbibresource{references.bib}

%rotating allows text to be printed sideways
%multirow allows tables with cells spanning more than one row
% Remove this if not required
%\usepackage{rotating,multirow}


\begin{document}
\author{Gregory Cartwright}
\title{}
%\maketitle
\section*{Introduction}

\subsection*{Context}
The author is an advanced nurse practitioner (ANP) in a community Trust.
The Trust has community hospitals which have wards for inpatient rehabilitation and
medical stepdown. Their are 12 wards accross 8 community hospitals. Each ward has
an ANP who works on the ward to provide medical management of the patients during 
the hours of Monday to Friday, 0900 to 1730. Outside of these hours, medical care 
is provided by the out of hours GP service. 

Most patients are admitted to the community hospital wards for either rehabilitation
or medical stepdown. The majority of these patients come following an acute admission
where they have been stabilised medically but are often deconditioned as a result
of acute illness and are not ready to go home. At this stage their needs include 
ongoing medcial treatment and monitoring and further assessments and treament such 
as physiotherapy and occupational therapy to prepare them for discharge home.

Some patients are admitted directly from home because they have medical or rehabilitation
needs that cannot be met at home but do not require an admission to an acute setting.
A minority of patients are admitted for paliative care.

When patients are admitted to one of these wards they have a comprehensive geriatric 
assessment (CGA) \parencite{bgs:14} performed by the multi-disciplinary team (MDT).
An international meta-analysis found that, when compared with general medical care,
CGA was effective at keeping older people alive and living in their own homes at
twelve months post admission with a number needed to treat of 33 \parencite{ellis:11}.

\subsection*{}

\clearpage
\printbibliography[prenote=needsfixing]
\end{document}
