\documentclass[12pt,a4paper,oneside,titlepage]{article}
%titlepage prints the title on it's own page in article class

%doublespace my text
\renewcommand{\baselinestretch}{1.5}

%reduce the hyphenation of words
\sloppy

%prints a small space between paragraphs and removes the indenting of the first line
\setlength{\parskip}{2ex plus0.5ex minus 0.2ex}
\setlength{\parindent}{0em}

\usepackage[utf8]{inputenc}
%csquotes - provides multilingual quoting - keeps babel happy
\usepackage{csquotes}
%babel required for biblatex to work. Specifying british last make it the language in use
%\usepackage[english,british]{babel}
\usepackage[british]{babel}
%\usepackage[T1]{fontenc}

\usepackage[style=authoryear,backend=biber,
	giveninits=true,
	uniquename=init,
	maxbibnames=99,
	language=british]{biblatex}

\defbibnote{needsfixing}{\emph{(this formatting needs looking at -- not currently to LSBU standards.
For example, need to include names of all the authors.)}}
% Add my bibliography file here
\addbibresource{references.bib}

%rotating allows text to be printed sideways
%multirow allows tables with cells spanning more than one row
% Remove this if not required
%\usepackage{rotating,multirow}


\begin{document}
\author{Gregory Cartwright}
\title{Proposal}
%\maketitle
\section*{Introduction}

\subsection*{Background}
The author is an advanced nurse practitioner (ANP) in a community Trust.
The Trust has community hospitals which have wards for inpatient rehabilitation and
medical stepdown. Their are 12 wards accross 8 community hospitals. Each ward has
an ANP who works on the ward to provide medical management of the patients during 
the hours of Monday to Friday, 0900 to 1730. Outside of these hours, medical care 
is provided by the out of hours GP service. When patients are admitted to one of
these wards they have a comprehensive geriatric assessment (CGA) \parencite{bgs:14} 
performed by the multi-disciplinary team (MDT).

Sometimes patients get admitted to the acute hospital during the out of hours period due to a 
deterioration in their condition.
These events are reviewed retrospectively. Sometimes the ANP feels that, 
due to their subjective level of frailty, transfer 
to an acute hospital was probably not in the patient's best interest and that it
would have been better for them if this deterioration had been managed locally.
This makes me feel that, prior to the event, I should have discussed with the patient and
their family about developing a treament escalation plan (TEP).

There is ongoing debate about the definition of frailty 
\parencite{shamliyan:13}, however it is a condition where the maintenance of homeostasis becomes vulnerable to 
small stressors and is acknowledged to be a large problem amongst the ageing population 
\parencite{clegg:13}.
Whilst the \textcite{bgs:14} advise that frailty is assessed whenever an older person
comes into contact with a healthcare professional, it is not routinely assessed 
in these community hospital inpatients. This raises the question that if all these 
patients were assessed for frailty, and their level of frailty was known to the ANP,
would this make the ANP aware of how vulnerable their homeostais was.
This could then trigger the ANP to consider the pre-emtive planning of escalation
of treament and discussing a TEP with the patient and their family. Indeed,
it is recommended that all patients with frailty have such a plan considered
\parencite{bgs:14}.

The multiple defitions of frailty can be split into two groups: those based on accumulation 
of deficits or co-morbidities and those that consider the phenotype \parencite{shamliyan:13}. 
The category of definiton informs how frailty is assessed and various tools have been developed
to asses frailty based on each of these models \parencite{clegg:13}.
The Clinical Frailty scale (CFS) \parencite{dalhousie:15} uses clinical judgment 
to assess frailty and has been vaildated
against tools that use an alternative model as a predictor of adverse outcomes 
of frailty \parencite{rockwood:05}. It may also be useful in predicting
inpatient mortality \parencite{wallis:15}.
Use of the CFS to
assess frailty is recommended only after a CGA has been performed \parencite{bgs:14}, therefore
it would be appropriate to use it in the community hospital patients after they
have been assessed by the MDT.

\subsection*{Literature review}
\emph{The literature review needs to go here.}

\subsection*{Aims and objectives}
The main aim of this study is to evaluate whether routine assessment of
frailty can add value for community hospital inpatients.
More specifically, can assessing patients with the Clinical Frailty Scale (CFS)
help increase the rate of 
treatment escalation planning for patients with severe frailty. 

\subsubsection*{Objectives}
\begin{enumerate}
\item Ascertain how many patients from the Trust have acute admissions and assess 
	whether the community hospital ANP team, and possibly the MDT, 
	feel that there is a  problem with patients 
	having unplanned acute admissions when, in rerospect, the risks involved
	outweigh the benefits. \emph{Do I need to think about other stakeholders, eg management
		and OOH?}
\item Assess the frailty of the Trust community hospital inpatient population
	and what proportion of these are severely frail.
	These patients should have had a treatment escalation plan considered.
\item Assess what proportion of community hospital inpatients have had consideration
	of treament escalation planning. 
	\emph{Question: Should I only do this for those that are severely frail?}
\item Explore what are the percieved barriers to planning treatment escalation
	with severely frail patients.
\item Plan and possibly implement an intervention that would address a percieved
	barrier to increase the rate of treament escalation planning for severely frail
	community hospital inpatients.


\end{enumerate}

\subsection*{Research design}

All admissions to acute hospital from the Tust community hospital wards are reviewed
so the number of such admissions is available within the organisation.

There are two objectives that lend themselves to being addressed with a questionnaire.
The first of these is to assess whether there is a problem with severely frail 
patients being admitted to an acute hospital when the risks outweigh the benefits.
The second part is what are the barriers that prevent escalation planning 
being considered.

The questionnaire would be circulated to all qualified healthcare
professionals involved with inpatient care. 

\emph{I need to discuss questionnaire design.}

Completion of the questionnaires would be encouraged by members of the ANP team 
on each ward. (I need to work out whether paper or online questionnaires would be best.)

The CFS grades frailty on a scale from 1, very fit, to 9, terminally ill. The
process used to assess frailty with the CFS is based on clinical judgment of the 
patient's level of independence with activities of living. Reviewing the
assessments that patients have had by the MDT will provide this information
and thus allow a CFS score to be ascertained for each patient. The Trust is currently 
rolling out an electronic patient record (EPR) for the community hospital wards and 
about half of the wards are now on this system. To assess the 
frailty of the Trust inpatient population it is proposed that the records of all
current patients who have an EPR are reviewed and their frailty is calculated.



It is suspected that most patients will have a CFS score of 7 of more. The above process 
will confirm or refute this. If we assume that this is the case, then it seems
that many of these patients should have a TEP. This leads to the question: 
How many patients have a TEP that can easily be idetified?

\begin{itemize}

\item Assess how many patients on each ward have a TEP. This should be identifiable on 
	the electronic handover, so it could be done remotely.
\end{itemize}

The author suspects that this number will be much smaller than the number of patients 
who have a high frailty score. This then leads to the question: 
What barriers exist to prevent patients having a TEP?


The next stage would be to identify the main barrier and design an intervention to
address this. If, for example, a barrier is that we don't consider frailty level and
therefore do not think about a TEP, then the intervention could be to add a CFS score
to the electronic handover that is used at the daily board round to stimulate MDT
discussion of possible need of a TEP. This would probably have a financial cost, so
may not be achievable. Adding a CFS score to the electronic patient record (EPR) would 
probably be easier to achieve, but may not stimulate the same level of MDT discussion.
Another potential issue would be that, as the EPR is currently in a process of staged 
rollout, there are only about half the wards on EPR, the remainder remain using paper notes.

The intervention could be introduced, and then the audit for number of patients with 
a TEP could be repeated to see if the intervention has improved the situation. If there 
is no improvment then the problems and barriers could be re-examined to look for another
intervention that might help.

\clearpage
\printbibliography[prenote=needsfixing]
\end{document}
