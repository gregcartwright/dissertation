\documentclass[12pt,a4paper,oneside,titlepage]{article}
%titlepage prints the title on it's own page in article class

%doublespace my text
\renewcommand{\baselinestretch}{1.5}

%reduce the hyphenation of words
\sloppy

%prints a small space between paragraphs and removes the indenting of the first line
\setlength{\parskip}{2ex plus0.5ex minus 0.2ex}
\setlength{\parindent}{0em}

%csquotes - provides multilingual quoting - keeps babel happy
\usepackage{csquotes}
%babel required for biblatex to work. Specifying british last make it the language in use
%\usepackage[english,british]{babel}
\usepackage[british]{babel}

\usepackage[style=authoryear,backend=biber,
	giveninits=true,
	uniquename=init,
	language=british]{biblatex}

\defbibnote{needsfixing}{\emph{(this formatting needs looking at -- not currently to LSBU standards.
For example, need to include names of all the authors.)}}
% Add my bibliography file here
\addbibresource{references.bib}

%rotating allows text to be printed sideways
%multirow allows tables with cells spanning more than one row
% Remove this if not required
%\usepackage{rotating,multirow}


\begin{document}
\author{Gregory Cartwright}
\title{Proposal}
%\maketitle
\section*{Introduction}

\subsection*{Background}
The author is an Advanced Nurse Practitioner (ANP) in a community Trust.
The Trust has community hospitals which have wards for inpatient rehabilitation and
medical stepdown. Their are 12 wards accross 8 community hospitals. Each ward has
an ANP who works on the ward to provide medical management of the patients during 
the hours of monday to friday, 0900 to 1730. Outside of these hours, medical care 
is provided by the out of hours GP service.

Sometimes patients get admitted to the acute hospital during out of hours period due to a 
deterioration in their condition.
These events are reviewed retrospectively. Sometimes the ANP feels that, 
due to their subjective level of frailty, transfer 
to an acute hospital was probably not in the patient's best interest and that it
would have been better for them if this deterioration had been managed locally.
This makes me feel that, prior to the event, I should have discussed with them and
their family about developing a treament escalation plan (TEP).

Frailty is acknowledged to be a large problem amongst the ageing population 
\parencite{clegg:13}, and makes maintenance of homeostais vulnerable to small stressors.
Whilst the \textcite{bgs:14} advise that frailty is assessed whenever an older person
comes into contact with a healthcare professional, it is not routinely assessed 
in these community hospital inpatients. This raises the question that if all these 
patients were assessed for frailty, and their level of frailty was known to the ANP,
would this make the ANP aware of how vulnerable their homeostais was.
This could then trigger the ANP to consider the preemtive planning of escalation
of treament and discussing a TEP with the patient and their family. Indeed,
it is recommended that all patients with frailty have such a plan considered
\parencite{bgs:14}.

\subsection*{What is the size of the problem?}
The clinical frailty scale (CFS) uses clinical judgment to assess level of frailty
\parencite{dalhousie:15}. It has been validated as a tool for predicting adverse
outcomes due to frailty \parencite{rockwood:05} and it may be useful in predicting
inpatient mortality \parencite{wallis:15}.

\begin{itemize}
\item Look at the data to find how many patients accross all the community wards in the
    Trust are readmitted. This data is collected monthly so will be easy to access.
\item Assess the frailty of 10 patients from each ward. Use CFS for this.
	The ANP on each ward would do this.
\end{itemize}

The CFS grades frailty on a scale from 1, very fit, to 9, terminally ill.
It is suspected that most patients will have a CFS score of 7 of more. The above process 
will confirm or refute this. If we assume that this is the case, then it seems
that many of these patients should have a TEP. This leads to the question: 
How many patients have a TEP that can easily be idetified?

\begin{itemize}

\item Assess how many patients on each ward have a TEP. This should be identifiable on 
	the electronic handover, so it could be done remotely.
\end{itemize}

The author suspects that this number will be much smaller than the number of patients 
who have a high frailty score. This then leads to the question: 
What barriers exist to prevent patients having a TEP?



\clearpage
\printbibliography[prenote=needsfixing]
\end{document}
