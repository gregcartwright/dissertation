\documentclass
[
	12pt,
	a4paper,
	oneside,
%	titlepage
]{report}
%titlepage prints the title on it's own page in article class

\usepackage[utf8]{inputenc}
\pagestyle{myheadings}
\markright{Gregory Cartwright --- 3403548}

%doublespace my text
%\renewcommand{\baselinestretch}{1.5}

%reduce the hyphenation of words
\sloppy

%prints a small space between paragraphs and removes the indenting of the first line
\setlength{\parskip}{2ex plus0.5ex minus 0.2ex}
\setlength{\parindent}{0em}

%csquotes - provides multilingual quoting - keeps babel happy
\usepackage{csquotes}
%babel required for biblatex to work. Specifying british last make it the language in use
%\usepackage[english,british]{babel}
\usepackage[british]{babel}

\usepackage[style=authoryear,backend=biber,
	giveninits=true,
	dateabbrev=false,
	uniquename=init,
	citestyle=authoryear,
	dashed=false,
	maxcitenames=2,
	maxbibnames=99,
	sorting=nyt,
	language=british]{biblatex}

%Add a comma between name and year in citations
\renewcommand*{\nameyeardelim}{\addcomma\space}
%put double space between entries in the references
\setlength\bibitemsep{2\itemsep}

%remove hanging indent from the reference list
\setlength\bibhang{0em}

%Ensure all names in reference list are formatted as Surname, I.
\DeclareNameAlias{default}{last-first}
\DeclareNameAlias{sortname}{last-first}

%Code to format journal article volume and issue as v (i)
\renewbibmacro*{volume+number+eid}{
	\printfield{volume}
	\setunit*{\addnbspace}
	\printfield{number}
	\setunit{\addcomma\space}
	\printfield{eid}
}
\DeclareFieldFormat[article]{number}{\mkbibparens{#1}}

%Code to format the 'Accessed day month year' in the references
\DeclareFieldFormat{urldate}{%
	[Accessed \thefield{urlday}\addspace%
  \mkbibmonth{\thefield{urlmonth}}\addspace%
  \thefield{urlyear}]}
  
%Code to format the 'Available from:' in the URL in references
\DeclareFieldFormat{url}{\bibstring{urlfrom}\addcolon\space\url{#1}}
\DefineBibliographyStrings{british}{
	urlfrom = {Available from}
}

%Command to STOP the references section and everything after having it's own header.
% The first for reports, second for articles
\defbibheading{bibintoc}[\refname]{\chapter{#1}}
%\defbibheading{bibliography}[\refname]{\section*{#1}}

\usepackage{graphicx}


% Add my bibliography file here
\addbibresource{references.bib}

%rotating allows text to be printed sideways
%multirow allows tables with cells spanning more than one row
% Remove this if not required
%\usepackage{rotating,multirow}

\begin{document}
\author{Gregory Cartwright\\
	Student number 3403548\\
	MSc Advanced Nurse Practitioner dissertation
}
\title{Does level of frailty influence advance care planning 
	for community hospital inpatients?
}

\maketitle

\begin{abstract}
The abstract goes here.
\end{abstract}

\section*{Acknowledgments}
Graeme Pettifer - data collection, sounding board.
Simon Conroy - sounding board.
Caroline Barclay - sounding board.
Mandy Cooper - general support.
Jonny Dexter - data collection.
Karen Plowman - data collection.
Mandy Cooper - general support.
Ruth Tandy - data collection.
Lynn MacDiarmid - data collection.

\tableofcontents 

\chapter{Introduction}

\section{Background}

\subsection{Population factors}

It is known that, due to improvements in lifestyle and better healtcare, 
people are living longer \parencite{nao:08,ons:17}. Subsequently the population of 
the UK is getting older and this trend is forecast to continue.
In 2016 18\% of the population was over 65 years old. This is expected to be
23.9\% by 2036 \parencite{ons:17}.


People are living longer but with more comorbities. The older people are, the more
likely they are to have multiple chronic diseases. In a large scale study in
Scotland, \textcite{barnett:12} found that the percentage of the population with multi morbidity was around 
65\% in those aged 65 to 84, rising to 81\% in those aged 85 or over. A simulation 
to project multimorbidity over the next 20 years found that the number of people
aged 65 or over that have two or more diseases is likely to increase by 86\% and the
increase of those with four or more disease is forecast to be 157\%
\parencite{kingston:18}. 

\subsection{Introducing frailty}

Multimorbidity is linked to frailty which can be viewed as a state where 
the function and resilience of multiple body systems is impaired \parencite{woo:14}. 
The prevalance of frailty is therefore likely to increase as the population
gets older \parencite{sharp:13}. Indeed, a review of all hospital admissions in 
England found an increase in inpatient frailty over an eight year period 
\parencite{soong:15}.

There are varying definitions of frailty \parencite{soong:15}, however it is 
widely agreed to be a condition where the maintenance of homoeostasis 
becomes vulnerable to  small stressors \parencite{vellas:16}. Examples of such 
stressors include changes in environment and minor illness. The consequences of 
exposure to these are wide ranging and include delirium, significant reduction 
in mobility, falls, increased dependency, non-specific failure to thrive and death 
\parencite{bgs:14,oliver:14,vellas:16}.

\subsection{Implications of frailty}

A person with frailty who becomes ill is therefore exposed to many risks. If
they are subsequently admitted to an acute hospital then an added stressor of a 
change in enviroment is added, further increasing the likelihood of them developing
new problems. Indeed, frailty combined with acute illness carries a high risk 
of death.

Frailty is known as the biggest cause of mortality in older people 
\parencite{gill:10}. As a person becomes increasingly frail, they move towards 
the end of life, even 
though they do not have a definitive terminal diagnosis. This transition is
often not recognised by the healthcare team, possibly because it is a gradual process
compared to something like cancer entering a terminal phase when all curative
options are exhausted \parencite{oliver:14}. Dying with frailty is common.
Frail older people who have no terminal diagnosis account for about 40\% of 
deaths and only a quarter of deaths are from malignant disease \parencite{sharp:13}.

\subsection{Risks and benefits}
\label{sec:risk-ben}

Two of the four ethical prinicples of healthcare beneficence and nonmaleficence
\parencite{beauchampChildress:01}.
When making decisions on how to proceed with a patient's care and treatment
it is therefore necessary to establish what the likely benefits and possibly 
harms are from the proposed actions. It is also necessary to gain an insight 
into the likelyhood of these benefits and harms, given the overall situation that
the patient is in at this time. This requires a knowledge of the patient's history
and their current illness.


When a person is in this phase, the likely benefits of acute admission are small.
The risks involved with admitting a person in this situation
to hospital are great.  
Indeed admission to acute hospital for people with frailty is associated with
poor outcomes including death \parencite{silver:12, wallis:15}. Risks and 
benefits have to be assessed and 
weighed up and Sometimes the risks of such an undertaking acutally
outweigh the intended benefits and the best outcome for the patient is that 
they are not admitted. 

Decision making around this situation is complex 
The patient is often not able to contribute to this process due to changes in their
mental capacity due to delirium. If the assessing clinician does not know the 
patient and their situation and history then this decision making is even more 
difficult. Often in such situations clinicans consider that the safest option is to 
admit the patient to an acute setting.

This is set against the year on year increase in the number of emergency hospital 
admissions in England, predominantly consisting of older people and multiple 
national initiatives which are in place to combat this \parencite{nao:18}.

\subsection{Advance Care Planning}

If decling condition due to frailty can be identified then discsussions with the 
patient and thier family can be had whilst they are not acutely unwell. These 
discussions should include what is important to the patient at this stage of their
life and plans can be made as to what should be done in particular circumstances
if their health deteriorates. This is known as advance care planning (ACP).

If an advance care plan is in place when a person becomes acutely unwell then it
can guide the decision making process for the clinican so that the most appropriate
action can be taken at that time, taking into account plans that have been made 
at at time when the person's health was more stable. This hopefully produces an
outcome for the patient and their family that is least risky and least distressing.

This sounds ideal, however the \textcite{silver:12} reports that end of life care 
in older people with frailty
is something that is often not adequately considered. It has been suggested that 
end of life care services in the UK are aimed at those with malignant 
disease \parencite{sharp:13}, futhermore it seems that people with frailty 
are often not involved in planning their 
end-of-life care \textcite{oliver:14}. partly due to the aforementioned assertion 
that, in this group, entering an end of life phase is frequently not recognised 
\textcite{wallington:16} and
estimating prognosis is more problematic \parencite{silver:12}.
Frailty should be considered at the centre of older people's healthcare to guide
evidence based treatment \parencite{woo:14}.

\section{Local context}

\subsection{Practice environment}

The author is an advanced nurse practitioner (ANP) in a community Trust.
The Trust has community hospitals which have wards for inpatient rehabilitation and
medical step-down. There are twelve wards across eight community hospitals. Each ward has
an ANP who works on the ward to provide medical management of the patients during 
the hours of Monday to Friday, 0900 to 1730. Outside of these hours, medical care 
is provided by the out of hours (OOH) general practitioner (GP) service. 

Most patients are admitted to the community hospital wards for either rehabilitation
or medical step-down. The majority of these patients come following an acute admission
where they have been stabilised medically but are often deconditioned as a result
of acute illness and are not ready to go home. At this stage their needs include 
ongoing medical treatment and monitoring and further assessments and treatment such 
as physiotherapy and occupational therapy to prepare them for discharge home.

Some patients are admitted directly from home because they have medical or rehabilitation
needs that cannot be met at home but don't require an admission to an acute hospital.
A minority of patients are admitted for palliative care.

When patients are admitted to one of these wards they have a comprehensive geriatric 
assessment (CGA) \parencite{bgs:14}. This is a multi-faceted diagnostic process
performed by the multi-disciplinary team (MDT) to produce a plan for treatment 
and follow-up.
An international meta-analysis found that, when compared with general medical care,
CGA was effective at keeping older people alive and living in their own homes at
twelve months post admission with a number needed to treat of 33 \parencite{ellis:11}.

\subsection{Local frailty}

The author expects that many of the patients admitted to the wards have frailty.
The numbers of patients with frailty is not known, however the patients are assessed 
for frailty and the patients admitted are old. During the last financial year the
average age was 81 and 49\% of patients admitted were aged 85 or over. Many have 
care needs as seen above.

Although the patients are usually clinically stable when they arrive on the ward,
their condition can deteriorate. This is usually managed in the community hospital
ward environment. The patients can have investigations including blood tests, plain
x-rays and electrocardiograms (ECG) usually without leaving the community hospital.
They can also receive treatments such as intravenous (IV) fluids, IV antibiotics
and even blood transfusions.

There are times when a patient deteriorates such that optimum mangagement of their
acute condition can only be delivered in an acute inpatient setting. This is when 
the sort of decision making discussed above in section~\ref{sec:risk-ben} has
to be tackled. The situation that the patient is in has to be evaluated. The 
expected benefits of the proposed acute admission have to be ascertained for the
individual patient and these need to be balanced against the risks posed by a transfer
to an acute setting for that person in that situation. There are many factors that 
will influence this process. Some of these will not be immediately obvious to a 
clinician who is meeting the patient for the first time.

When a patient deteriorates during the OOH period they will be reviewed by a clinician
from the OOH GP service. This practitioner will be unlikely to know the patient,
the situation they are in and what their priorities in life are. As the patient is
acutely unwell at this point they may be unable to engage in objective and measured
discussions about how their care should proceed at this point. The OOH GP may also
not be fully aware of the capabilities of the community hospital. 

In an audit of acute hospital admissions from communtity hosopital,
\textcite{endacott:15} found that 55\% of these admissions happened OOH. 
The OOH clinician only has a partial picture of the patient's situation. It may 
be that they often will err on the side of caution and
admit patients to the acute setting when this course of action may not actually have
been in the best interest of the patient. This is reinforced by retrospective reviews
of all acute admissions of community hospital inpatients that are carried out by the 
ANP team. These often find that the admission could have been avoided. In the case
of particularly frail patients, the ANPs sometimes feel that transferring that
patient to the acute sector was probably not in the best interest of the patient. 
They then feel that they should have engaged that patient in advance care planning
earlier in their stay to limit inappropriate, unhelpful and possibly detrimental 
escaltion of their care to an acute setting.

\section{Research focus}

\subsection{Clarifying the problem}

There are patients in community hospital beds whose health deteriorates during
the OOH hours period. Some of these patients are very frail and because advance
care planning has not been undertaken with them they get transferred to an acute 
hospital when the net benefit of this to the patient is likely to be negative.
Some patients in the community hospital wards do have such advance care planning 
carried out and subsequently, when some of them suffer an acute deterioration in
their health, they avoid being transferred to acute hospital. This ensures a more
dignified outcome for them, eliminating the risk and distress of the emergency
department. Many of them recover.

Reducing the number of patients who get admitted to the emergency department would
also help to reduce the strain on that department. Ultimately this would be a
benefit to the local health economy.

Currently there is no formal criteria for which patients get advance care planning.
The decision to commence this process is made subjectively, based on assessment 
by the ANP or geriatrician. Most of these patients are likely to be old or frail
or both. The proportion of local patients who get advance care planning is not known,
but the systematic review conduced by \textcite{sharp:13} found that most older
people wanted to have such discussions, with some patients believing that it 
should be a routine proceeding.

It seems therefore that advance care planning should be happening more in community
hospitals. If there were more formal criteria to prompt consideration
of advance care planning would this increase how often if happens? Could their 
level of frailty act as a suitable criteria?

\subsection{What research is there in this area?}

A quick literature search shows that There is a paucity of literature looking 
into frailty scoring in community hospitals.
A search of the LSBU library journals collection for ``community hospital frailty''
did not find any articles examining any factor that could prompt ACP.

Changing the focus of the search terms to ``frailty advance care planning''
was more fruitful.
An intervention to involve care home residents and community hospital patients
had a large uptake with more than half the participants taking up the option of
advance care planning that was offered to them \textcite{mcglade:17}. This 
study did not select people for ACP but offered it to all, however, given the 
setting of long term care, it can be assumed that the prevalence of frailty
amongst the participants was high. This study supports the feeling that many
older people with frailty would welcome ACP.

Whilst exploring avoidable admissions to acute hospital \textcite{mytton:12} 
suggests that avoiding unecessary admission depends on good decision making at 
the time. This is difficult when the person assessing the patient does not know
them or their history. If we can prompt more consideration of ACP then this 
should assist with informing the decision making process at the time of 
deterioration.

The \textcite{bgs:14} advise that frailty should be used as a prompt to
commence individualised planning of future care. This should include plans
for urgent and emergency situations.

This project aims to look at the link between the patient cohort in the CH
welcoming ACP, frailty being an adverse prognostic factor in acute illness and
ACP being explored with appropriate patients to ensure that, in urgent and 
emergency situations, the patient gets what is best for them.

%Argue why my research needs to be done.
%What will my research explore? 

%Why do I want to reseach this area?

%Describe what I will be exploring. - ??? have I done this already ???





\section{Overall research aim and individual objectives}

The overall aim of this study is to examine if there is a relationship 
between
level of frailty of community hospital in-patients and whether advance care planning
happens and what other factors might influence this process.

It has been seen that patients with frailty should be identified and this 
information should be used by clinicians to explore advance care planning with 
This should subsequently help prevent unplanned acute admissions that are 
unlikely to be of benefit to the patient and are likely to cause them distress 
and harm.

The level of frailty amongst the patient cohort in the local CH is not currently
known. Some of these patients do undergo ACP during their inpatient stay, but what 
influences when this is done is not explicitly known. Frailty may be a factor
in this decision making process, but it is likely that there are patient 
related elements that are also influencial.

% ***** objectives
Therefore the objectives of this study are to:
\begin{enumerate}
\item	Ascertain the prevalence of different levels of frailty within the local 
		community hospital population, and within these levels identify 
		how many do not have an ACP before admission.\label{obj:prevalence}
\item	Determine which factors, namely frailty score, age, gender and
		presenting complaint, influence whether or not an ACP is 
		completed during the inpatient stay.\label{obj:association}
\item	Formulate local recommendations for practice to help reduce unhelpful
		acute hospital admissions for people with frailty through more 
		effective preemptive planning of treatment escalation.
		\label{obj:recommend}
\end{enumerate}
% ***** objectives

Objective~\ref{obj:prevalence} will provide an insight into the demographics
of the patient cohort. Given that frailty should prompt consideration of ACP
it will provide information about the proportion of the patient population in
whom this should be considered. Patients who have an ACP prior to their admission
have already been through the ACP process, and it is assumed that the ACP will be 
noted by the CH team and translated into an appropriate plan for their CH stay.
With these patients it is therefore not appropriate to consider what factors 
influence ACP as it has already been undertaken, so these patients will not
be considered in the work for objective~\ref{obj:association}.

We know that some patients have advance care planning whilst they are in the
community hospital but what is it about the patients that prompts the team to
consider it for some patients and not for others?
Objective~\ref{obj:association} will explore this question by examining the CH
journey of those patients 
who arrive there without an ACP. It will look at the facets of age, frailty, 
gender and presenting complaint in an attempt to see whether any of these are
influencual factors in the process of deciding whether the advance care planning
will be considered by the team for that patient.

The results of this exploration will be discussed along with the guidance that 
identification of frailty should prompt conisderation of ACP and the knowledge
that many older people with frailty welcome such a process, to provide some 
criteria that can then be used by the CH clinicians to help them idetify 
patients who will benefit from discussion about their future care at a time 
when their health is relatively stable.

%\section{Outline reseach methods and timescales}

%\section{Value of this research}

\chapter{Literature review}

\section{Introduction}

This literature review will firstly look to establish what frailty is, how it
manifests and explore startegies that exist to assess frailty. With this 
background knowledge the research objectives can be tackled. 
Objective~\ref{obj:prevalence} relates to the prevalence of frailty in the
local CH population. Therefore the incidence of frailty at an international
and national level will be examined to give the reader an idea of what the
size of the issue of frailty is locally.

To understand the need for advance care planning in the context of frailty
the consequneces of frailty will be explored with a particular focus on the
combination of frailty and acute illness or acute deterioraion in health. 
It is situations such as this which has been identified as the problem area
in the local CH.

The question of how to address and manage frailty will be considered. In 
particualr the focus will be on what is recommended in the literature as 
strategies for determining the factors, in frailty, that can help to prompt
ACP in an effort to inform the reader about the background for research 
objective~\ref{obj:association}.

The driving force behind that objective is the benefits of ACP for people with 
frailty, therefore the proposed and observed boons of ACP will be briefly 
visited.

It is hoped that this will provide a good background to the area of research
and the intent is that a rationale for the need for this study will be 
exhibited. To clarify the situation, a summary of the issues that emerge from
the literature review is presented followed by a justificaion for the
empirical work that has been carried out in this study.

\section{Defining frailty}

To consider frailty as a criteria by which to measure a person it is necessary
to understand what is meant by it. The agreement in the literature is that ther
are many different definitions of frailty
\parencite{ensrud:08,rockwood:05,conroy:09}, and that whilst it is linked to
ageing, dysability and multimorbidity it is a distinct syndrome
\parencite{fried:01,conroy:09}. In recent yearsthe definitions have, to some
extent, coalesced and the overarching guidance for frailty in the 
UK is provided by the \textcite{bgs:14}. They link frailty as a concept to the 
ageing process, but clarify that it is not an inevitable consequence of growing
old by informing us that, in those aged 85 and over, it affects between 25\% 
and 50\% of people. 

The definition provided by \textcite{bgs:14} is wide: where ``multiple body 
systems gradually lose
their inbuilt reserves''. This does not explain the effect that frailty has upon
a person. \textcite{clegg:13} clarify this by describing how frailty makes an
individual more susceptible to adverse outcomes as a result of small stressors.
A much earlier definition is provided by \textcite{fried:01}. They
start by reporting that mulitple definitions of frailty existed at the time, 
but that the emerging definition is that of a situation where, like the later
definitio, multiple body systems have lowered ability to fight stressors. What
they go on to do is to provide and validate a model of frailty based on a phenotype
with observable characteristics. The utility of this is that it allows 
clinicians to apply the model and identify people with frailty: a person who has
3 or more of the attributes in phenotype is considered to be frail. 

Use of this model relies on a knowledge of the characteristics of the population,
making it unsuitable for clinical use \parencite{ensrud:08}.
Furthermore, \textcite{martin:08} notes a pitfall of the phenotype systme is 
that it fails to
consider cognitive or psychosocial factors that contribute to frailty.
Another drawback is how it categoriese people: not frail,
pre-frail or frail. This does not allow for levels of frailty, lumping anyone with
frailty into one group. As frailty is a consquence of the relaitive failure of
multiple physiological systems it is clearly a dynamic state. It is more complex
than a simple binary system. Depnding on the various abilities or impairment of
the many body systems the level of frailty will be an analogue quantity. People
with frailty will be somewhere on a continuum of frailty. \textcite{jones:05} 
addressed both these criticisms and proposed frailty is viewed as an accumulation of
impairments. They created a frailty index, with points for each impairment or
deficit; the more points someone has the frailer they are. This allows for a more 
fine grained straification of frailty.

The strategy used here is to grade frailty based on the outcomes of a CGA. Ten
functional domains are assesed and scored based on whether the person has a
problem in that domain: 0 points if there is no problem, 0.5 points for a small
problem, 1 point for a big problem. This gives possible scores from 0 to 10 
including half points. Problems with this approach can be seen. It is quite
time consuming to evaluate all 10 domains. Also there is an element of 
subjectivity: when does a small problem become a big problem? This is quantified
for six of the ten domains, however if the other four are assessed
wrongly then there is potiential for considerable error in the score. 

\textcite{jones:05} argue that their index is aligned with ther notion of a 
fitness-frailty continuum, and highlight that it is also a good model as frailty
is a multi-system deficit syndrome, reflected by their ten domains looking
across many systems. The practical clinical use of such a scale is questionned
by \textcite{rockwood:05} who suggest that it is too time consuming for such an
application. They assert that a better approach for clinical use is one based
on clinical judgement, taking into account data collected through history taking
and physical assessment. They go on to report how they developed and validated
such a tool: the clinical frailty scale (CFS). This scale varys from 1 (very fit)
to 8 (very severely frail) with point 9 being terminally ill 
(see Appendix~\ref{apx:cfs}).

It is accepted that different frailty assessment tools are suited to different
applications \parencite{ensrud:08,martin:08,romero-ortuno:16}: research, public
health planning and clinical practice. The CFS is recommended for use in urgent
care by the Acute Frailty Network (www.acutefrailtynetwork.org.uk) and is
used routinely in the local emergency department.

\section{Prevalence of frailty}
\label{sec:litrevprev}

Frailty is a state of multiple body system impairment to an extent that a
person's abiltiy to deal with relatively small insults is impaired. Tools have 
been developed and validated that are convenient to use in a clinical setting.
How prevalent is frailty? \textcite{collard:12} conducted a systematic review 
to assess levels of frailty in people aged 65 and over. They found that over
10\% of these were frail, rising to at least 26\% of those aged over 85 years.
Although they examined 21 cohorts, 61,500 participants, their favoured tool for 
assess frailty was the phenotype model of \textcite{fried:01}. Their reationale
for this is that it is a tool favoured by reseachers. As discussed 
above, however, it is not an ideal clinical tool and does not consider cognitive or
psychosocial issues that contribute to frailty. Therefore the study may have 
underestimated the prevalance of frailty.

Prevalence is also esimated by \textcite{clegg:13} at between 25\% and 50\% of
those aged 85 and over, however the sources they cite are a subset of those
used by \textcite{collard:12} so the figures are not likely to be any more
accurate. It is however these same figures that are quoted by the
\textcite{bgs:14}, lending credence to this assessment of the situation. It
is also supported by a more recent French study which found the prevalence of
frailty in the over 65 age group to be 9.3\%, again useing the phenotype model
\parencite{cossec:16}.

\section{Implications and consequences of frailty}

Frailty is a syndrome that has increasing incedenc with age, leaving the person
more suscebitble to minor stressors or insults. \textcite{collard:12} suggests
that frailty is linked to an increased risk of death. Indeed,
an examination
of all bereavements of adults in England that were not due to accident, homicide or suicide for 
a four month period in 2012 was 
carried out \parencite{ons:13}. It found that the proportion of deaths
that were not due to cancer or any cardiovascular disease (CVD) was 42\%. 
In the over 80
age group this was 80\%. The most recent edition of this survey from 2015 found
that the overall proportion of non-cancer and non-CVD deaths was slightly higher 
at 46\% \parencite{ons:16}, but did not provide a breakdown by age group. They 
did however report that 60\% of their sample were aged over 80.

How many of these deaths were due to frailty is not known, however 
a Canadian study that examined all deaths in Alberta found that frailty was the
cause of 30\% of mortality \parencite{fassbender:09}. This is anecdotally
supported by \textcite{silver:12} who report that many older people are admitted 
to hospital only to die within hours, and that
this is particularly true for people admitted during the out of hours period.

Qualifying this susceptability
has been attempted. 
A chinese study looked at the rather specific cohort of patients aged over 65
who were admitted with acute coronary syndrome (ACS) in a Chinese hospital for 
a six month period \parencite{kang:15}. They measured frailty with CFS and 
found that frailty was independently associated with an increased risk of 
inpatient death, and also significantly increased the risk of readmission and 
3-month mortality for those who survived to hospital discharge.

In a recent study of emergency admission of patients over 
75 years old through a full 12 month period at a single site, frailty as
measured by CFS was explored against mortality, 30-day readmission rate and 
length of stay (LOS)
\parencite{wallis:15}.
These criteria make this study relevant to the local situation, and they found 
that CFS was a predictor of all three of these variables. The increased risk of
30-day readmission with frailty was supported by \textcite{kahlon:15} whose 
Canadian study sed CFS to assess frailty of adult general medical patients. 
They also identified that frailty was associated with an increased risk of death
within 30 days post discharge.

\section{Recommendations for people with frailty}

Frailty is a multi-dimensional syndrome that has severe consequences,
specifically when combined with acute illness. Tools are available that facilitate
screening for it. Indeed, national guidance in the UK is that older patients 
should be screened for frailty at every contact with a health professional 
\parencite{bgs:14}. Once frailty has been identified, there is consensus that
an MDT approach should be employed for more in-depth assessment of the 
problems of the individual \parencite{vellas:16}. The comprehensive geriatric
assessment (CGA) is such a strategy that is evidence based. \textcite{ellis:11}
performed An international 
meta-analysis and found that, when compared with general medical care,
CGA was effective at keeping older people alive and living in their own homes at
twelve months post admission with a number needed to treat of 33 \parencite{ellis:11}.
Consequently, assessment with the CGA is advocated by the \textcite{bgs:14}
and The King's Fund \parencite{oliver:14}.

We know that frailty carries paarticular risk for the person. Subsequent to their
assessment of models of frailty, 
\textcite{martin:08} assert that clinicians can use frailty assessment to place 
greater confidence
in calculation of risk and benefit for patients. This argument is elaborated by 
a suggestion that if frailty is used to indentify those at risk of multiple 
hospital admissions then plans can be made for those people to be better
supported in the community. This suggests that frailty should prompt advance care
planning. This viewpoint is supported by 
\textcite{hunt:16}, who when examining how to prevent unhelpful acute admissions
for older poeple advocates that frailty should be used to guide personalisd 
care planning.

Further weight is added to this argument by \textcite{kang:15}, who having 
found that frailty in the 
context of acute coronary syndrome is associated
with poor outcomes, \textcite{kang:15} recommend that a high CFS should
trigger the consideration of escalation pathways \parencite{kang:15}.
This supports the recommendations of \textcite{silver:12} who assert that over-investigation
and unnecessary interventions in the frail elderly population are costly to both the
individual and the health economy. They go on to advise that in such patients, 
their preferences for their future care should be ascertained early. \textcite{oliver:14} 
reinforce this by highlighting the importance of gathering this information
before the person loses the capacity to make decisions about how their care should
progress. The later work of \textcite{romero-ortuno:16} adds weight to this argument.
Having identified the increased risk associated with frailty and hospital admission, 
they recommend that frailty generally should trigger personalised planning of
care and it's escalation.

\section{What are the benfits of advance care planning?}

Advance care planning discussions are often not easy to undertake 
\parencite{taylor:17}, however \textcite{sharp:13} conducted a systematic
review of attitudes towards these discussions in frail people. They 
demonstrated that most older people would rather have such discussions
whilst they were relatively well, viewing not having such discussions
as a risk to them not getting their future care and treatment in a way
that they would want.

Medical care is dominmated by a desire to cure \parencite{taylor:17}. In the 
context of frailty and acute illness we have seen that this is not always 
possible. This is often not easily recognised by clinicians, particulalry
at a time of crisis, such as an acute deterioration in health. This can lead
to a default position of subjecting the patient to an admission to an acute
hospital, which may cause more harm than benefit. A transparently written 
systematic review exploring non-benefical treatment in the last six months
of life found between a third and a half of patients in this position being
subjected to non-beneficial imaging and blood tests and between 11\% and
75\% of patients recieving non-beneficial drugs \parencite{cardona:16}.
This is concerning and needs to be tackled. \textcite{taylor:17} postulates
that the medical principle of first doing no harm has got lost, and that
creation of an ACP helps patient choice be realised.

\textcite{waird:16} emphasizes that the purpose of ACP is not necessarily
to prevent all hospital admissions, but to consider under which 
circumstances an admission would be beneficial and which it would be of
no net benefit.

\section{Emerging issues and why is there a need for empirical 
research?}

We have seen that frailty is a problem amongst the growing older population.
This causes increased vulnerabiltiy to these people and the risks involved with 
admission to acute hospital are great. This is compounded by the co-existence
of acute illness at this juncture. It is difficult to ascertain prognosis with 
frailty, but frailty can signify that a life is fragile. Older people with 
frailty often welcome their healthcare staff engaging them in discussions 
surrounding plans for their futurte care. If the healthcare team is not fully
aware of the severity of their prognosis, how likely is it that the patient
has a good understanding of this? 

Undertaking such discussions with the patients allows their hopes and fears to
be explored. It helps build a shared understanding of what is important to
them and how best these aspirations can be achieved. Plans can then be 
developed to facilitate this becoming reality. Such discussions do happen in
community hospitals, but the author has been unable to find any literature
that explores this phenomenon and factors that surround it.

\chapter{Research Methods}
 
\section{Introduction}

Some estimation of the prevalence of frailty has been garnered in 
section~\ref{sec:litrevprev} of the literature review. However, to fully
investigate research objective~\ref{obj:prevalence} and assess the incedence
and level of frailty and the proportion of patients who do not have an
existing ACP on admission to the CH it will be necessaary to collect some
data. Examining patient factors and how they relate to the incedence of
the ACP process within the CH (objective~\ref{obj:association}) will
also require the collection of empirical data.

\section{Research strategy}

Research objective~\ref{obj:prevalence} is asking a question about numbers
of patients in particular categories. This requires a quantitative approach
\parencite{biggam:15}.
Objective~\ref{obj:association} is a little more complex. The question it is 
asking is around what influences the decision making process for individual
patients as to whether or not the advance care planning is undertaken. One 
approach to this could be to ask why the decisions are made. This would 
entail exploring the decision making process with those staff who are 
involved, namely the ANPs and geriatricians. A qualitative approach would
be suitable to this strategy \parencite{jolley:13}, and could involve 
focus groups or interviews with staff. Benefits of this plan would be that it 
would facilitate an understanding of the beliefs and motivations of the 
decision makers \parencite{parahoo:14}. A problem with this approach for the
context of this dissertation is that it is inherently time consuming
\parencite{jolley:13}.

Despite the positive aspects of a qualitative approach to addressing 
objective~\ref{obj:association}, this plan may not consider the conept of 
frailty
as a conrtributary factor. The staff may not be aware of the CFS score
that a patient has when considering ACP, and a major part of the objective
is to ascertain whether there is any relationship between frailty and ACP.
It therefore seems more appropriate to assess the level of frailty for each
patient and then look to see if they were involved in ACP. This involves
collection of numerical data and analysis of relationships between variables, 
so is suited to a quantitative approach \parencite{parahoo:14}. As the objective
is to also consider other possible influencial varuiables, these can also
be collected and analysed at the same time.

To gain a more complete understanding of the situation it would be useful to 
have the quantitative and qualitative data described above. This would provide
both statistical information about relationships between ACP and patient
variables and also an insight into the actual decision making processes
that are employed. This process is known as triangulation as it uses different
sources of data to gain multiple perspectives on an issue 
\parencite{biggam:15}. It was felt that to adopt such a tack would involve a 
considerable amount of time for a sole researcher and that the timescale of 
this dissertation was unfortunately insufficient to permit such a mixed 
methods approach.

This study is looking at independent patient variables and at the outcome or
dependent variable 
of ACP. It is not iterfering with how care is delivered; we are not 
manipulating one
variable and observing another vairable: there is no experiment.
It is simply examining the statistical relationship between variables and
therefore the 
study will be correlational \parencite{field:09}. It will be a retrospective 
observational 
cross-sectional study: the case-notes of discharged patients will be reviewed.
To achieve objective~\ref{obj:prevalence} data will be obtained by reviewing 
the case-notes of patients, examining the initial MDT assessments to ascertain 
CFS score and whether an advance care plan was in place prior to admission. The
CFS is designed to be used after a CGA has been carried out \parencite{bgs:14}
and an example of it being used from retrospective observation of casenotes is
provided by \textcite{subbe:13} using it for a national audit.
For objective~\ref{obj:association} the entire case-notes of patients 
will be reviewed to ascertain whether advance care planning was considered 
during the stay and to obtain other information that may have influenced
advance care planning.

\section{Sampling}

To gain a representative picture of the patient population it is necessary
to select a sample of patient casenotes to review. Calculation of a sample size 
was performed using the website www.surveysystem.com/sscalc. To provide a confidence
level of 0.05 and a confidence interval of 90\% it was decided that a sample 
size of 100 would be required. To improve this power it was decided that a sample
of 150 participants would be possible in the time given.

As for selectng the sample there are various methods, which can be grouped as 
probability and non-probability sampling \parencite{parahoo:14}. Probability 
sampling, also known as random sampling \parencite{biggam:15}, involves a 
sampling system where every
member of the study population has a non-zero chance of being in the sample.
This type of sample is considered to help reduce study bias.

The setting for the study is group of 12 
communtity hospital wards. At the time of the study the group was part way 
through a process of becoming ``paper light'': changing from traditional paper
MDT notes to an electronic patient record (EPR). This transition was being made
one ward at a time, and most of the wards had moved to the EPR. This meant that 
the patient records could be accessed remotely, from another site within the 
Trust over over virtual private network (VPN). To enable quick collection of data
it was decided to collect data from only wards which were on EPR. The whole of the 
patient stay would have to be examined, so EPR would beed to have been in place
on the ward for the duration of their stay.

Taking this into consideration the following convenience sampling method was 
utilised. Wards were included that had been on EPR for at least three months
prior to the first Monday of Febraury 2018 -- 5 February 2018. Then all patients
discharged on this date onwards were included until there was a sample of 150
patients. The average length of stay is 20.4 days. This sampling strategy 
meant that the bed base being sampled from was 148 of the total 214 beds: 69\%.

This is a type of non-probabilty method of sampling, but facilitated rapid
collection of data, which given the timescale of the dissertation, was a big
deciding factor.

\section{Data collection}

Data collection was conducted based on the sampling method described above.
The EPR was searched for all discharges on or after 5 February 2018. Wards
that were not on EPR still had their admissions and discharges recorded on the
EPR system and subsequently came up in the search. These patients were 
manually removed from the list and searching continued until there were 150
in the sample.

Each patient was then individually looked up in the EPR. The data was
collected in a Microsoft Excel spreadsheet table (see appendix~\ref{apx:tool}).

Research objective~
\ref{obj:prevalence} requires assessment as to whether the patient has a
pre-existing ACP when they were admitted to the CH ward. 
Objective~\ref{obj:association} required collection of patient age, gender,
presenting complaint and CFS. 

For this objective The record also had to be searched to se if advance care
planning had been considered at any time during the patient admission. To 
achieve this the record was searched for the following terms:

\begin{itemize}
\item acute
\item escalation
\item advance care plan
\item ACP
\item deterioration
\item ceiling
\end{itemize}

The search term ``ceiling'' was included as it was found that some practitioners
had documented outcome of advance care planning as having agreed ``ceiling'' of
the level of care for the patient. For example ``this patient should not be
transferred to an acute hospital, community hospital is their ceiling of care.''

Where one of these terms was found, the record was read in context to assess if 
escalation planning or advance care planning was being considered. If this
was considered at least once during the admission then this was recorded as 
``YES"
in the data collection tool. Otherwise ``NO" was recorded in that column.

The author received some help with data collection from colleagues.

The presenting complaint was the reason that the patient was originally 
admitted to hoospital. The EPR records were carefully read to determine this,
and it was recorded in the table. To facilitate statistical analysis, the 
presenting complaint was subsequently grouped or coded:
\begin{enumerate}
\setcounter{enumi}{-1}
\item Fall
\item Infection
\item Malignancy
\item Stroke
\item Gastro-intestinal
\item Musculoskeletal
\item Other neurological
\item Cardiovascular
\item Other
\end{enumerate}

A fall is usually multifactorial in cause \parencite{silver:12} and therefore
does not usually have a single cause that fits into one category. For this 
reason falls were given their own code.
These patients were admissted during the winter season and influenza was the
presenting complaint of some of them. When this was the case it was coded as
infection. Two of the wards that were studied were specialist stroke 
rehabilitation wards, so there was a significant number of patients admitted
with stroke. If general decline in mobibilty with no single identifed cause
was the presenting complaint then this was coded as musculoskeletal.
Confusion with no specific cause is coded as other neurological.



\section{Framework for data anlysis}

Once the data was collected in the Excel table, this was imported into 
Statistical Package for the Social Sciences (SPSS) version 21 which was then
used for statistical analysis of the data.

For analysis it was helpful to group ages and CFS. Ages of subjects were 
grouped as follows:
\begin{itemize}
	\item less than 50
	\item 50 to 65
	\item 65 to 70
	\item 70 to 75
 	\item 75 to 80
	\item 80 to 85
 	\item 85 to 90
 	\item 90 to 95
 	\item 95 to 100
 	\item over 100
\end{itemize}

Frailty was categorised by grouping CFS:
\begin{itemize}
\label{ref:cfs-grouping}
	\item mild frailty: CFS 1 to 5
	\item moderate frailty: CFS 6
	\item Severe frailty: CFS 7 to 9
\end{itemize}

Objective~\ref{obj:prevalence} involves looking at the levels of frailty of
subjects in the sample. To illustrate this the study sample has been 
displayed in tables showing the numbers and percentages for CFS score, 
frailty category, age, gender and presenting complaint. Within each
of these the numbers and percentages who had a pre-existing ACP
on admission to community hospital are also shown.

To assist with understanding of this the distribution of the study sample 
has been presented graphically as follows.
Histograms are used to show distribution of participants by individual 
CFS score and age group.  
To illustrate the distribution by category of frailty, by gender and by presenting 
complaint, pie charts are used. Within these graphs, different shading shows
the proportion that had a pre-existing ACP on admission to the community hospital.
Descriptive statistics such as the mean and median age and the standard deviation
are also calculated and presented to help gain an understanding of the 
demographic of the sample.

% **** descriptinve stuff for obj 2
Objective~\ref{obj:association} is about exploring what happens around ACP
during the partient's CH stay. Therefore, prior to any analysis for this 
objective, all patients who already had an ACP on arrival at the CH were
removed from the data.
The susbsequent initial presentation of data for this objective was
graphical. Bar charts show are used to show the relative frequency with which 
advance care planning
was carried out for each level of CFS and group of frailty and for gender and
presenting complaint. Histogram is used to illustrate this for patient age.

The rest of this section is incomplete and will be completed once I've written
the results chapter.

The data is either ordinal or nominal, so for advanced or inferential 
statistics 

%*****
%
% come back to this when I've worked it out!!!
%
%****

%%%%%%
% Variables
% =========
%
% Independent
% ----------
% + age -			discrete	-> continuous
% + age  group - 		interval	-> continuous 
% + Gender - 			binary 		-> categorical
% + CFS - 			ordinal		-> categorical
% + Frailty level		ordinal		-> categorical
% + Presenting complaint - 	nominal 	-> categorical
%
% Dependent
% ---------
% +ACP happened			binary		-> categorical
%%%%%%

% Is age normally distributed?
%
% Can get SPSS to draw normal distribution on histogram and also to do a
% P-P plot which plots expected cumulative probability against
% observed probability. Done this - looks pretty good.
%
% Also can get it show all extra stuff on the frequenceis menu
% - see Field page 137.
% or use the Kolmogorov–Smirnov test and Shapiro–Wilk test - Field p.144

Chi-squared ($\chi^2$) requires expected frequeny in each cell to be at least 5 
\parencite[page 690]{field:09}. So where this was a problem I have used Fisher's
Exact test.

%%%%
% Statistical tests
% 
% Chi squared for CFS
% point-biserial icorrelations for age, age group
% Chi squared for gender
% ??? for presenting complaint
%
% Do I need to look at look at serial or semi-serial correlations?
% Do I need to compare correlations?
%%%%%

%Look at the statistics used by wallis(2015) to look at CFS and mortality.

\chapter{Results}

\section{Sample distribution}

The sample included 150 patients who had been discharged from CH. Two patients
were removed because they had only been in the ward overnight and therefore
there was insufficient data in their records to be able to determine their
CFS. This left a sample of 148 patients. 

Of these, 88 (59.5\%) were female and 60 (40.5\%) were male. The demographics 
and characteristics of the sample are shown in table~\ref{tab:dist-overall}.

\begin{table}[p]
\caption{Sample demographics and characteristics}
\label{tab:dist-overall}
\includegraphics[width=\textwidth,
	trim={1.5cm 4cm 2.5cm 2cm},
	clip,
	angle=90,
	scale=1.45]{media/dist-overall}
\end{table}

Patients in the sample group were relatively old: the mean age of patients was 
81 years with a median of 84 years. The youngest was 40 with the oldest being 
99. 
The initial presenting complaint was not evenly spread. Almost 80\% of the
were originally admitted with one of the top three complaints: Fall 37.8\%, 
infection 29.1\% and stroke 12.2\%. 

When viewed by CFS, the distribution shows a high level of frailty. The median 
CFS was 6 - moderately frail, with only 36.5\% of the sample being less frail
than this. Having said that, only just over 8\% of the sample had a CFS of 8
or 9, with 55\% of the sample having CFS of either 6 or 7 - moderate or
severe frailty. 

When frailty is grouped as described in section~\ref{ref:cfs-grouping} the 
distribution is much more uniform with mild, moderate and severely frail
representing 36.5\%, 27.7\% and 35.8\% of the sample respectively.
See figure~\ref{fig:chart-dist-frailty-level}.
\label{sec:results-dist}

% Not sure that I need this graph?
\begin{figure}[ht]
\caption{Distribution by frailty level}
\label{fig:chart-dist-frailty-level}
\includegraphics[width=\textwidth,
	trim={2.5cm 14cm 2.5cm 2.5cm},
	clip]{media/chart-dist-frailty-level}
\end{figure}


Very few patients had a preexisting ACP on admission to CH. There were only 6
patients, 4.1\% of the sample.


\section{Advance care planning}

Out of the sample of 148, six had a pre-existing ACP on admission to CH. These 
were discounted from future analysis leaving a sample of 142. Of these 18 
(12.7\%) had ACP during the hospital stay. There were no patients that 
declined ACP. The statistics regarding association between the independent 
variables; CFS, frailty level, age, gender and presenting complaint; and the 
dependent variable of whether ACP happens is presented in 
table~\ref{tab:statistics}).

\begin{table}[ht]
\caption{Summary of staistical analysis}
\label{tab:statistics}
\includegraphics[width=\textwidth,
	trim={2.5cm 10cm 2.5cm 2.5cm},
	clip]{media/statistical-analysis}
\end{table}

\subsection{CFS}

The most common CFS score of patients 
who had ACP during CH stay was CFS=7. Out of the 18 patients that
underwent ACP in the CH, 7 of them (38.9\%) had a CFS of 7. This represents 
17.5\% of the patients with this CFS. The CFS score which had the highest 
proportion of it's patients having ACP was CFS=8 where 3 out of the 4 patients 
with this CFS underwent ACP ($p=0.011$). 


\subsection{Frailty level}

When frailty is grouped into mild, moderate and severe, we saw in 
scetion~\ref{sec:results-dist} that the
sample was relatively evenly distributed. 
Whilst about one-third of the sample were severely frail, two-thirds of all 
patients who underwent ACP were in this frailty category. One-quarter of
the patients with severe frailty underwent ACP, compared with mild and moderately
frailty of which 7.5\% and 4.9\% respectively had ACP ($p=0.007$).
Figure~\ref{fig:chart-frailty-level-acp} illustrates this.

\begin{figure}[ht]
\caption{ACP and level of frailty}
\label{fig:chart-frailty-level-acp}
\includegraphics[
	width=\textwidth,
	trim={2.5cm 2.5cm 2.5cm 2.5cm},
	clip]{media/graph-frailty-level-acp}
	% This chart had changed in SPSS !!!!
\end{figure}

\subsection{Gender}

There was no significant association between gender and ACP: 12.7\% for
females and 12.5\% for males ($p=1.0$).

\subsection{Age}

Age was associated with ACP ($p=0.016$). No-one under the age of 66 
received ACP. Half of the 18 patients aged between 91 and 95 years underwent
ACP but only 6.2\% of those in the 86 to 90 year old category got ACP. This
asociation is not monotonic and is illustrated in 
figure~\ref{fig:chart-age-acp}.

\subsection{Presenting complaint}

There was a significant association between presenting complaint and ACP
($p=0.034$). Patients that presented with a primary diagnosis of stroke or a GI 
or MSK disorder did
not have ACP. All the other categories had proportions of patients that had ACP
ranging from 10.9\% (fall) to 50\% (both malignancy and CVS).

\begin{figure}[ht]
\caption{}
\label{fig:chart-age-acp}
\includegraphics[
	width=\textwidth,
	trim={2.5cm 1.5cm 2.5cm 1.5cm},
	clip]{media/chart-age-ACP}
\end{figure}

\chapter{Discussion}

\section{Limitations}

\subsection{Methods}

A mixed methods approach would have provided richer data and would probably
have produced more useful findings.

Non-probablistic sampling limits generalisability (reference)

Excluding certain wards means that some members of the
population had zero chance of being in the sample,
further compromising generalisability. Talk about how this
and seasonal nature of the ssmple, limits generalisability
to the Trust, not just to the resesrch community.
Conducting the study at a different time of year or across
a 12 month period may yield different results. In defence
of this strategy, it has enabled a quick snapshot to be
generated which will hopefully identify issues that can be
addressed. The audit could easily be revised or repeated 
over a longer time period. At this juncture all the wards 
will have been on EPR for a substantial period of time so
allowing a probablistic sampling mathod to be employed.

% Read about study reliability. comment on how reliable
% the study is likely to be and what steps were taken to
% maximise this.

Reliability: to speed up data collection help was given to the author by a few
colleagues. Much of the data is unambiguous: age and gender, and, arguably
whether ACP happened. The situation of CFS does however require interpretation
of patient casenotes, 
% *** Look up inter-rater reliability of CFS

Bias: the author was working outside the Trust at the time 
of data collection so was not involved in the care of any
of the patients sampled. This hopefully allowed objective
intrrpretation of the notes and associated decision making.

Thinking about how to code the initial presenting complaint: may need to think about
categories (Malignancy, fall, infection, stroke, delirium) And about multiiple problems
(1 diagnosis, 2 diagnoses etc) - multiple diagnoses are 
common amongst older people (reference).

\subsection{Results}

I found it difficult to differentiate between CFS of 7 and 8 from casenotes.

There may have been some subjectivity with the CFS scoring - is CFS validated 
for use from casenotes?

The study was carried out during the winter period and used the conveniemce
sampling method, so only considered people who were in hospital at this time 
of year. Hospital caseloads vary with the seasons (ref) in terms of the 
presenting complaints and the pressures associated with the season. Therefore
the sampling technique is likely to have compromised how representative the
study is of the system across the whole year.

The study was limited to 8 of the 12 wards in the Trust. Two of these wards
are stroke wards and the four wards that were not included in the study were 
general wards. Therefore the proportion of patients in the sample
who were admitted following stroke is likely to have been higher than if the 
study was to be repeated taking a sample from accross all the wards. There may 
well be different attitudes towards advance care planning for patients who 
primary reason for admission was stroke when compared to the patients who
were admited for another reason.

Stroke is potentially very life changing. Anecdotally, (find a reference) 
people who need
inpatient stroke rehabilitation have a wide variety of premorbid functional 
abilities and therefore varied premorbid frailty. These variations will
affect the feelings of the patient and their attitiudes to future care and will
likely influence the decision making around ACP that the team make. Another
possible factor is that the consultants with responsibilty for the stroke 
patients are specialist stroke physicians and will have slightly different 
thought processes to the geriatricians who work on the remaining wards.
and treatment.

\chapter{Conclusion}
The conclusion.

\printbibliography[heading=bibintoc]

\clearpage
%\begin{appendix}

\appendix
\chapter{Clinical Frailty Scale}
\label{apx:cfs}
\includegraphics[width=\textwidth]{CFS}

\chapter{Data collection tool}
\label{apx:tool}
\includegraphics[width=\textwidth]{dataCollection}


%\end{appendix}

\end{document}
