\documentclass
[
	12pt,
	a4paper,
	oneside,
%	titlepage
]{article}
%titlepage prints the title on it's own page in article class

\pagestyle{myheadings}
\markright{Gregory Cartwright --- 3403548 --- dissertation proposal}

%doublespace my text
\renewcommand{\baselinestretch}{1.5}

%reduce the hyphenation of words
\sloppy

%prints a small space between paragraphs and removes the indenting of the first line
\setlength{\parskip}{2ex plus0.5ex minus 0.2ex}
\setlength{\parindent}{0em}

%csquotes - provides multilingual quoting - keeps babel happy
\usepackage{csquotes}
%babel required for biblatex to work. Specifying british last make it the language in use
%\usepackage[english,british]{babel}
\usepackage[british]{babel}

\usepackage[style=authoryear,backend=biber,
	giveninits=true,
	dateabbrev=false,
	uniquename=init,
	citestyle=authoryear,
	dashed=false,
	maxcitenames=2,
	maxbibnames=99,
	sorting=nyt,
	language=british]{biblatex}

%Add a comma between name and year in citations
\renewcommand*{\nameyeardelim}{\addcomma\space}
%put double space between entries in the references
\setlength\bibitemsep{2\itemsep}

%remove hanging indent from the reference list
\setlength\bibhang{0em}

%Ensure all names in reference list are formatted as Surname, I.
\DeclareNameAlias{default}{last-first}
\DeclareNameAlias{sortname}{last-first}

%Code to format journal article volume and issue as v (i)
\renewbibmacro*{volume+number+eid}{
	\printfield{volume}
	\setunit*{\addnbspace}
	\printfield{number}
	\setunit{\addcomma\space}
	\printfield{eid}
}
\DeclareFieldFormat[article]{number}{\mkbibparens{#1}}

%Code to format the 'Accessed day month year' in the references
\DeclareFieldFormat{urldate}{%
	[Accessed \thefield{urlday}\addspace%
  \mkbibmonth{\thefield{urlmonth}}\addspace%
  \thefield{urlyear}]}
  
%Code to format the 'Available from:' in the URL in references
\DeclareFieldFormat{url}{\bibstring{urlfrom}\addcolon\space\url{#1}}
\DefineBibliographyStrings{british}{
	urlfrom = {Available from}
}

%Command to STOP the references section and everything after having it's own header.
\defbibheading{bibliography}[\refname]{\section*{#1}}

\usepackage{graphicx}


% Add my bibliography file here
\addbibresource{references.bib}

%rotating allows text to be printed sideways
%multirow allows tables with cells spanning more than one row
% Remove this if not required
%\usepackage{rotating,multirow}

\begin{document}
\author{Gregory Cartwright\\
	Student number 3403548\\
	MSc Advanced Nurse Practitioner dissertation
}
\title{Does level of frailty influence advance care planning 
	for community hospital inpatients?
}

\maketitle

\begin{abstract}
The abstract goes here.
\end{abstract}

\section{Introduction}

\subsection{Background}

It is widely accepted that, due to improvements in lifestyle and better healtcare, 
people are living longer \parencite{nao:08,ons:17}. Subsequently the population of 
the UK is getting older and this trend is forecast to continue.
In 2016 18\% of the population was over 65 years old. This is expected to be
23.9\% by 2036 \parencite{ons:17}.


People are living longer but with more comorbities. The older people are, the more
likely they are to have multiple chronic diseases. In a large scale study in
Scotland, \textcite{barnett:12} found that the percentage of the population with multi morbidity was around 
65\% in those aged 65 to 84, rising to 81\% in those aged 85 or over. A simulation 
to project multimorbidity over the next 20 years found that the number of people
aged 65 or over with two or more diseases is likely to increase by 86\% and the
increase of those with four or more disease is forecast to be 157\%
\parencite{kingston:18}. Frailty can be viewed as a state where multiple body 
systems are impaired and therefore is linked to multimorbidity
\parencite{woo:14}. It's prevalance is likely to increase as the population
gets older \parencite{sharp:13}. Indeed, a review of all hospital admissions in 
England found an increase in inpatient frailty over an eight year period 
\parencite{soong:15}.

There are varying definitions of frailty \parencite{soong:15}, however it is 
widely agreed to be a condition where the maintenance of homoeostasis 
becomes vulnerable to  small stressors \parencite{vellas:16}. Examples of such 
stressors include changes in environment and minor illness. The consequences of 
exposure to these include delirium, significant reduction in mobility,
falls, increased dependency, non-specific failure to thrive and death 
\parencite{bgs:14,oliver:14,vellas:16}.

This makes them more vulnerable to small changes.

Frailty should be at the centre of older people's healthcare to guide evidence
based treatment \parencite{woo:14}.

Acute hospital admission carries risks.

These are greater for people with frailty.

Frailty combined with acute illness carries a high risk of death. 
The \textcite{silver:12} reports that end of life care in older people with frailty
is something that is often not adequately considered. It has been suggested that 
end of life care services in the UK are aimed at those with malignant 
disease \parencite{sharp:13}.
This is supported by \textcite{oliver:14} who assert that people with frailty 
are often not involved in planning their 
end-of-life care. They suggest that the reasons for this include factors such as
the trajectory, with frailty often being a more gradual decline without sudden 
landmark moments. This contrasts with conditions such as terminal cancer where there 
is a defining transition to an end-of-life phase. This can mean that entering such a
phase is not recognised and therefore planning is overlooked. 


End of life phase is often not identified in patients with frailty. 
Frail older people who have no terminal diagnosis account for about 40\% of 
deaths and only a quarter of deaths are from malignant disease \parencite{sharp:13}.

Benefits of acute admission are reduced for people with frailty.

Risk benefit analysis - favour of not admitting and managing in current location.

Community hospital can actually many treatments.

Decision making during OOH is difficult.

Therefore patients often get admitted.

This should be planned in advance to ensure that patients get most appropriate 
and beneficial treatment. 

The article \textcite{sharp:13} contains lots of good
stuff on advance care planning in frailty, for example, most older people wanted
discussions about end of life earlier rather than later. Also some doctors found
having these discussions  more difficult in frailty rather than with a 
definitive terminal diagnosis.

\subsection{Research focus}
What research is there in this area? 
Is there a lack of research? 
Argue why my research needs to be done.
What will my research explore? 

Why do I want to reseach this area?

Describe what I will be exploring.

\subsection{Overall research aim and individual objectives}

\subsection{Outline reseach methods and timescales}

\subsection{Value of this research}

\section{Limitations}
I found it difficult to differentiate between CFS of 7 and 8 from casenotes.

\section{Acknowledgments}
Graeme Pettifer - data collection, sounding board.
Simon Conroy - sounding board.
Caroline Barclay - sounding board.
Mandy Cooper - general support.
Jonny Dexter - data collection.
Karen Plowman - data collection.
Mandy Cooper - general support.
Ruth Tandy - data collection.
Lynn MacDiarmid - data collection.

\section{Research methods}
Had to add ``ceiling'' to list of search terms.

\clearpage
\printbibliography
\end{document}
