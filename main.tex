\documentclass
[
	12pt,
	a4paper,
	oneside,
%	titlepage
]{report}
%titlepage prints the title on it's own page in article class

\pagestyle{myheadings}
\markright{Gregory Cartwright --- 3403548 --- dissertation proposal}

%doublespace my text
\renewcommand{\baselinestretch}{1.5}

%reduce the hyphenation of words
\sloppy

%prints a small space between paragraphs and removes the indenting of the first line
\setlength{\parskip}{2ex plus0.5ex minus 0.2ex}
\setlength{\parindent}{0em}

%csquotes - provides multilingual quoting - keeps babel happy
\usepackage{csquotes}
%babel required for biblatex to work. Specifying british last make it the language in use
%\usepackage[english,british]{babel}
\usepackage[british]{babel}

\usepackage[style=authoryear,backend=biber,
	giveninits=true,
	dateabbrev=false,
	uniquename=init,
	citestyle=authoryear,
	dashed=false,
	maxcitenames=2,
	maxbibnames=99,
	sorting=nyt,
	language=british]{biblatex}

%Add a comma between name and year in citations
\renewcommand*{\nameyeardelim}{\addcomma\space}
%put double space between entries in the references
\setlength\bibitemsep{2\itemsep}

%remove hanging indent from the reference list
\setlength\bibhang{0em}

%Ensure all names in reference list are formatted as Surname, I.
\DeclareNameAlias{default}{last-first}
\DeclareNameAlias{sortname}{last-first}

%Code to format journal article volume and issue as v (i)
\renewbibmacro*{volume+number+eid}{
	\printfield{volume}
	\setunit*{\addnbspace}
	\printfield{number}
	\setunit{\addcomma\space}
	\printfield{eid}
}
\DeclareFieldFormat[article]{number}{\mkbibparens{#1}}

%Code to format the 'Accessed day month year' in the references
\DeclareFieldFormat{urldate}{%
	[Accessed \thefield{urlday}\addspace%
  \mkbibmonth{\thefield{urlmonth}}\addspace%
  \thefield{urlyear}]}
  
%Code to format the 'Available from:' in the URL in references
\DeclareFieldFormat{url}{\bibstring{urlfrom}\addcolon\space\url{#1}}
\DefineBibliographyStrings{british}{
	urlfrom = {Available from}
}

%Command to STOP the references section and everything after having it's own header.
\defbibheading{bibliography}[\refname]{\section*{#1}}

\usepackage{graphicx}


% Add my bibliography file here
\addbibresource{references.bib}

%rotating allows text to be printed sideways
%multirow allows tables with cells spanning more than one row
% Remove this if not required
%\usepackage{rotating,multirow}

\begin{document}
\author{Gregory Cartwright\\
	Student number 3403548\\
	MSc Advanced Nurse Practitioner dissertation
}
\title{Does level of frailty influence advance care planning 
	for community hospital inpatients?
}

\maketitle

\begin{abstract}
The abstract goes here.
\end{abstract}

\section*{Acknowledgments}
Graeme Pettifer - data collection, sounding board.
Simon Conroy - sounding board.
Caroline Barclay - sounding board.
Mandy Cooper - general support.
Jonny Dexter - data collection.
Karen Plowman - data collection.
Mandy Cooper - general support.
Ruth Tandy - data collection.
Lynn MacDiarmid - data collection.

\tableofcontents 

\chapter{Introduction}

\section{Background}

\subsection{Population factors}

It is known that, due to improvements in lifestyle and better healtcare, 
people are living longer \parencite{nao:08,ons:17}. Subsequently the population of 
the UK is getting older and this trend is forecast to continue.
In 2016 18\% of the population was over 65 years old. This is expected to be
23.9\% by 2036 \parencite{ons:17}.


People are living longer but with more comorbities. The older people are, the more
likely they are to have multiple chronic diseases. In a large scale study in
Scotland, \textcite{barnett:12} found that the percentage of the population with multi morbidity was around 
65\% in those aged 65 to 84, rising to 81\% in those aged 85 or over. A simulation 
to project multimorbidity over the next 20 years found that the number of people
aged 65 or over that have two or more diseases is likely to increase by 86\% and the
increase of those with four or more disease is forecast to be 157\%
\parencite{kingston:18}. 

\subsection{Introducing frailty}

Multimorbidity is linked to frailty which can be viewed as a state where 
the function and resilience of multiple body systems is impaired \parencite{woo:14}. 
The prevalance of frailty is therefore likely to increase as the population
gets older \parencite{sharp:13}. Indeed, a review of all hospital admissions in 
England found an increase in inpatient frailty over an eight year period 
\parencite{soong:15}.

There are varying definitions of frailty \parencite{soong:15}, however it is 
widely agreed to be a condition where the maintenance of homoeostasis 
becomes vulnerable to  small stressors \parencite{vellas:16}. Examples of such 
stressors include changes in environment and minor illness. The consequences of 
exposure to these are wide ranging and include delirium, significant reduction 
in mobility, falls, increased dependency, non-specific failure to thrive and death 
\parencite{bgs:14,oliver:14,vellas:16}.

\subsection{Implications of frailty}

A person with frailty who becomes ill is therefore exposed to many risks. If
they are subsequently admitted to an acute hospital then an added stressor of a 
change in enviroment is added, further increasing the likelihood of them developing
new problems. Indeed, frailty combined with acute illness carries a high risk 
of death. 

As a person becomes increasingly frail, they move towards the end of life, even 
though they do not have a definitive terminal diagnosis. This transition is
often not recognised by the healthcare team, possibly because it is a gradual process
compared to something like cancer entering a terminal phase when all curative
options are exhausted \parencite{oliver:14}. Dying with frailty is common.
Frail older people who have no terminal diagnosis account for about 40\% of 
deaths and only a quarter of deaths are from malignant disease \parencite{sharp:13}.

\subsection{Risks and benefits}
\label{sec:risk-ben}

When a person is in this phase, the likely benefits of acute admission are small.
The risks involved with admitting a person in this situation
to hospital are great.  
Indeed admission to acute hospital for people with frailty is associated with
poor outcomes including death \parencite{silver:12, wallis:15}. Risks and 
benefits have to be assessed and 
weighed up and Sometimes the risks of such an undertaking acutally
outweigh the intended benefits and the best outcome for the patient is that 
they are not admitted. 

Decision making around this situation is complex 
The patient is often not able to contribute to this process due to changes in their
mental capacity due to delirium. If the assessing clinician does not know the 
patient and their situation and history then this decision making is even more 
difficult. Often in such situations clinicans consider that the safest option is to 
admit the patient to an acute setting.

\subsection{Advance Care Planning}

If decling condition due to frailty can be identified then discsussions with the 
patient and thier family can be had whilst they are not acutely unwell. These 
discussions should include what is important to the patient at this stage of their
life and plans can be made as to what should be done in particular circumstances
if their health deteriorates. This is known as advance care planning (ACP).

If an advance care plan is in place when a person becomes acutely unwell then it
can guide the decision making process for the clinican so that the most appropriate
action can be taken at that time, taking into account plans that have been made 
at at time when the person's health was more stable. This hopefully produces an
outcome for the patient and their family that is least risky and least distressing.

This sounds ideal, however the \textcite{silver:12} reports that end of life care 
in older people with frailty
is something that is often not adequately considered. It has been suggested that 
end of life care services in the UK are aimed at those with malignant 
disease \parencite{sharp:13}, futhermore it seems that people with frailty 
are often not involved in planning their 
end-of-life care \textcite{oliver:14}. partly due to the aforementioned assertion 
that entering an end of life phase is frequently not recognised in this group.
Frailty should be considered at the centre of older people's healthcare to guide
evidence based treatment \parencite{woo:14}.

\section{Local context}

\subsection{Practice environment}

The author is an advanced nurse practitioner (ANP) in a community Trust.
The Trust has community hospitals which have wards for inpatient rehabilitation and
medical step-down. There are twelve wards across eight community hospitals. Each ward has
an ANP who works on the ward to provide medical management of the patients during 
the hours of Monday to Friday, 0900 to 1730. Outside of these hours, medical care 
is provided by the out of hours (OOH) general practitioner (GP) service. 

Most patients are admitted to the community hospital wards for either rehabilitation
or medical step-down. The majority of these patients come following an acute admission
where they have been stabilised medically but are often deconditioned as a result
of acute illness and are not ready to go home. At this stage their needs include 
ongoing medical treatment and monitoring and further assessments and treatment such 
as physiotherapy and occupational therapy to prepare them for discharge home.

Some patients are admitted directly from home because they have medical or rehabilitation
needs that cannot be met at home but don't require an admission to an acute hospital.
A minority of patients are admitted for palliative care.

When patients are admitted to one of these wards they have a comprehensive geriatric 
assessment (CGA) \parencite{bgs:14}. This is a multi-faceted diagnostic process
performed by the multi-disciplinary team (MDT) to produce a plan for treatment 
and follow-up.
An international meta-analysis found that, when compared with general medical care,
CGA was effective at keeping older people alive and living in their own homes at
twelve months post admission with a number needed to treat of 33 \parencite{ellis:11}.

\subsection{Local frailty}

The author expects that many of the patients admitted to the wards have frailty.
The numbers of patients with frailty is not known, however the patients are assessed 
for frailty and the patients admitted are old. During the last financial year the
average age was 81 and 49\% of patients admitted were aged 85 or over. Many have 
care needs as seen above.

Although the patients are usually clinically stable when they arrive on the ward,
their condition can deteriorate. This is usually managed in the community hospital
ward environment. The patients can have investigations including blood tests, plain
x-rays and electrocardiograms (ECG) usually without leaving the community hospital.
They can also receive treatments such as intravenous (IV) fluids, IV antibiotics
and even blood transfusions.

There are times when a patient deteriorates such that optimum mangagement of their
acute condition can only be delivered in an acute inpatient setting. This is when 
the sort of decision making discussed above in section~\ref{sec:risk-ben} has
to be tackled. The situation that the patient is in has to be evaluated. The 
expected benefits of the proposed acute admission have to be ascertained for the
individual patient and these need to be balanced against the risks posed by a transfer
to an acute setting for that person in that situation. There are many factors that 
will influence this process. Some of these will not be immediately obvious to a 
clinician who is meeting the patient for the first time.

When a patient deteriorates during the OOH period they will be reviewed by a clinician
from the OOH GP service. This practitioner will be unlikely to know the patient,
the situation they are in and what their priorities in life are. As the patient is
acutely unwell at this point they may be unable to engage in objective and measured
discussions about how their care should proceed at this point. The OOH GP may also
not be fully aware of the capabilities of the community hospital. 

The author feels that often the OOH clinician will err on the side of caution and
admit patients to the acute setting when this course of action may not actually have
been in the best interest of the patient. This is reinforced by retrospective reviews
of all acute admissions of community hospital inpatients that are carried out by the 
ANP team. These often find that the admission could have been avoided. In the case
of particularly frail patients, the ANPs sometimes feel that transferring that
patient to the acute sector was probably not in the best interest of the patient. 
They then feel that they should have engaged that patient in advance care planning
earlier in their stay to limit inappropriate, unhelpful and possibly detrimental 
escaltion of their care to an acute setting.

Some patients in the community hospital wards do have such advance care planning 
carried out. 

******* HERE ******




This should be planned in advance to ensure that patients get most appropriate 
and beneficial treatment. 

The article \textcite{sharp:13} contains lots of good
stuff on advance care planning in frailty, for example, most older people wanted
discussions about end of life earlier rather than later. Also some doctors found
having these discussions  more difficult in frailty rather than with a 
definitive terminal diagnosis.

\section{Research focus}
What research is there in this area? 
Is there a lack of research? A quick literature search on ``frailty'' and
``community hospital'' returns very little (check this).
Argue why my research needs to be done.
What will my research explore? 

Why do I want to reseach this area?

Describe what I will be exploring.

\section{Overall research aim and individual objectives}

The objectives of this study are to:
\begin{enumerate}
\item	Ascertain the prevalence of different levels of frailty within the local 
		community hospital population, and within these levels identify 
		how many do not have an ACP before admission.\label{obj:prevalence}
\item	Determine which factors, namely frailty score, age, gender and
		presenting complaint, influence whether or not an ACP is 
		completed during the inpatient stay.\label{obj:association}
\item	Formulate local recommendations for practice to help reduce unhelpful
		acute hospital admissions for people with frailty through more effective
		preemptive planning of treatment escalation.
\end{enumerate}


\section{Outline reseach methods and timescales}

\section{Value of this research}


\chapter{Methodology}
 
Changed start date for discharges to 5 February to include more wards and reduce
the amount of sifting out of patients.

Had to add ``ceiling'' to list of search terms.

Thinking about how to code the initial presenting complaint: may need to think about
categories (Malignancy, fall, infection, stroke, delirium) And about multiiple problems
(1 diagnosis, 2 diagnoses etc)

Coding of presenting complaint. Describe how these were coded. Include that falls
have their own code, reduced mobility is coded as MSK unless there is another specific
cause. Confusion with no specific cause is coded as neuro.

Falls are usually multifactorial (need a reference for this), so have a coding 
of their own.

Flu was coded as infecion.

Chi-squared requires expected frequeny in each cell to be at least 5 (Andy Field,
page 690).

\chapter{Results}
I have to remove two patients because they were only in hospitalfor one night,
both being admitted to an acute hospital because they had deteriorated. They
had not been fully assessed and so there frailty could not be ascertained.

There was no one that declined an ACP.

\section{Limitations}
I found it difficult to differentiate between CFS of 7 and 8 from casenotes.

\chapter{Discussion}

\chapter{Conclusion}

\clearpage
\printbibliography
\end{document}
