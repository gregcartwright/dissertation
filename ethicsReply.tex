\documentclass
[
	12pt,
	a4paper,
	oneside
]{letter}
\usepackage[british]{babel}
\usepackage[british]{isodate}
\cleanlookdateon

\address{14 Baker Grove\\
	Ibstock\\
	Leicestershire\\
	LE67 6DB}
\name{Greg Cartwright}
\telephone{07977228838}

\begin{document}

\begin{letter}{Karen Sanders\\
		Chair of HSC Ethics Committee\\
		London South Bank University\\
		103 Borough Road\\
		London\\
		SE1 0AA}
\opening{Dear Karen,}

\bigskip
\textbf{18/A/26 -- Does level of frailty influence advance care planning 
	for community hospital inpatients? }
\bigskip

Thank you for your letter of 18th February 2018. I am grateful that you have 
approved my study subject to me addressing four issues that you have raised.

Firstly I would like to answer your second point. My intention with objective~2
is to examine the relationship between frailty and whether advance care
planning happens during the community hospital stay. There may be other factors 
that influence whether advance care
planning happens which may include the age of the patient, their gender and what 
their intial presenting complaint was. I will therefore collect all this data for
each patient so that I can analyse what might influence advance care planning. As
I am investigating what happens to patients during their stay in a community
hospital, if a patient has a pre-existing ACP when they are admitted to the 
community hospital, this will be recorded in the data collection tool.

Objective 2 should therefore be re-worded as follows:
	\begin{quote}
		Determine which factors, namely frailty score, age, gender and
		presenting complaint influence whether or not an ACP is 
		completed during the inpatient stay.
	\end{quote}

I will now discuss your first and third points together.

Frailty will be categorised by grouping CFS:
	\begin{itemize}
		\item mild frailty CFS 1 to 5
		\item moderate frailty: CFS 6
		\item Severe frailty: CFS 7 to 9
	\end{itemize}

Ages will be grouped as follows:
	\begin{itemize}
		\item less than 50
		\item 50 to 65
		\item 65 to 70
		\item 70 to 75
 		\item 75 to 80
		\item 80 to 85
 		\item 85 to 90
 		\item 90 to 95
 		\item 95 to 100
 		\item over 100
	\end{itemize}

Data analysis for objective 1: 

The study population will be preseted in tables showing the numbers and percentages
for CFS score, frailty category, age, gender and presenting complaint. Within each
of these I will also show the numbers and percentages who had a pre-existing ACP
on admission to community hospital.

The distribution of the study population will be presented graphically as follows.
To illustrate the distribution of participants by individual CFS score and age
histograms will be used. 

To illustrate the distribution by category of frailty, by gender and by presenting 
complaint, pie charts will be used.

Within each of these graphs, different colour or shading will be used to show
the proportion that had a pre-existing ACP on admission to the community hospital.

Data analysis for objective 2:

This objective is looking at what happens during the community hospital stay and 
thus relates only to people who did not have a pre-existing ACP when they
were admitted to the community hospital, therefore the analysis will only include these 
patients.

The data will initially be presented graphically. These graphs will aim to illustrate
any relationship between advance care planning being carried out during community
hospital stay and any factors that might have an 
influence on it: 
frailty, age, gender and initial presenting complaint.

A bar chart will be used to show the relative frequencies of advance care planning
for each level of CFS: 1 to 9. A further bar chart will show these frequencies with
CFS grouped into mild frailty (CFS of 1 to 5), moderate frailty (CFS of 6) and severe
frailty (CFS of 7 to 9). The y-axis in these charts will be the number of patients
for each category for whom advance care planning was carried out as a proportion of
the total number of patients in that category.  

Similar bar charts will be used to show the relative frequnecy of advance care
planning for gender, age group and presenting complaint.

Due to the data being ordinal and nominal in nature, non-parametric measures will
used to look for relationships. For this statistical analysis chi squared 
test or a Fisher's exact test will be applied to look for an association 
between advance care planning and the variables level of frailty, age, gender and 
presenting complaint.

In answer to your fourth point, I have provided a form that will be used to 
collect data in appendix B of my proposal. In this form the second, third and 
fourth columns will be "YES/NO" options. If the entry in the second column is 
"YES" then the remainder of the data will not be collected for that patient as 
they will not be included in the rest of the study.

I hope that this sufficient to satisfy the conditions that you raised. I look
forward to hearing from you further and to moving on the data collectin phase 
of my study.

\closing{Yours sincerely,}

\cc{Dr. Sharon Rees}

\end{letter}
\end{document}
